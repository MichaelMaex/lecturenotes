\lecture{5}{2020-03-05}{schemes}
The spectrum functor translates algebra into topology/geometry.
\begin{proposition}
	Let $R$ be a ring and $I$ be an ideal, $S \subset R$. 
	\begin{enumerate}
		\item $f:R \to R / I$ projection morphism. 
			Then $\spec f: \spec R / I \to \spec R$ is a homeomorphism onto $V(I)$. 
		\item $g: R \to R[s^{-1}]$ localisation morphism. 
			Then $\spec$ g is a homeomorphism onto $D(S)$. 
	\end{enumerate}
\end{proposition}
\begin{proof}
	We will only prove (1) as  (2) will be analogous.
	\begin{align*}
		\spec f: \spec R /I  &\longrightarrow \spec R \\
		\mathcal{P}  &\longmapsto f^{-1}(\mathcal{P} )
	,\end{align*}
	is an inclusion preserving bijection between $\spec R / I$ and $V(I)$, the set of prpime ideals in $R$ containing $I$. 
	The bijection \[
		\spec f: \spec R / I \to V(I)
	.\] 
	is a homeormophism:
	\begin{enumerate}
		\item continuous: yes
		\item closed: If $J$ is an ideal in $R / I$ then $(\spec f)(V(J)) =  V(f^{-1}(J)) \subset  V(I)$. 
	\end{enumerate}
\end{proof}

\subsection{Togological properties of $\spec R$} \label{sec:togological_properties_of_$\spec_r$}
\begin{recall}
	\begin{description}
	
	\item [quasicompact]	Recall that quasicompact means that every open cover has finite subcover. 
	Compact means quasicompact + Hausdorff. This the convention of the French school. 

\item [irreducible] Not a union of two closed strict subsets
\item [generic point of a topological space $X$ ] There is a point $x$ such that $\overline{\{x\} } = X$. 
\end{description}
\end{recall}

\begin{note}
	If $X$ has a generic point then it is irreducible.
\end{note}

\begin{proposition}
	Let $R$ be a ring, $I $ be an ideal. 
	\begin{enumerate}
		\item $\spec R$ is quasi compact. 
		\item $V(I)$ is irreducible if and only if $\sqrt{I} $ is prime. 
			In that case $\sqrt{I} $ is the unique generic point of $V(I)$. 
		\item In particular $\spec R$ is irreducible if and only if $n_R = \sqrt{(0)} $ is prime. 
			Then $n_R$ is the unique generic point of $\spec R$. 
	\end{enumerate}
\end{proposition}
\begin{corollary}
Recall that we had a inclusion reversing bijection between radical ideals and closed subsets in  $\spec R$.
This restricts to a bijection between prime ideals and irreducible closed subsets. \[
	I \to V(I)
.\] 	
\end{corollary}
\begin{proof}
	\begin{enumerate}
		\item The sets $D(S)$, $s \in R$ form a basis for the topology. Thus it suffices to show that every open cover of the form $\{D(s_i) \st i \in I\} $ with $s_i \in R$ has an open subcover.
			$\spec R = \bigcup_{i \in I}  D(s_i)$. 
			So No prime ideal in $R$ contains all the elementns $s_i$. 
			The set $\{ s_i \st i \in I\} $ generates the unit ideal $(1) = R$.
			This means that $1$ is a linear combination of a finite number of $s_i $'s. $1 = r_1 s_1 + \ldots  + r_n s_n$ for some $s_1, \ldots, s_n \in \{s_i | i \in I\} $. So $\spec R = D(s_1) \cup \ldots \cup D(s_n)$.
		\item \ltr First assume that $\sqrt{I}  $ is prime. 
			Then $\overline{\{\sqrt{I} \} } = V(\sqrt{I} ) = V(I)$. 
			So that $\sqrt{I}  $ is a generic pointn of $V(I)$ and $V(I)$ is irreducible.
			If $J$ is another generic point of $V(I)$ then $V(J) = \overline{\{J\} } = V(I)$. So $J = \sqrt{I} $ ($J$ is prime).

			\rtl Conversely, assume that $V(I) = V(\sqrt{I} )$ is irreducible. 
			We may assume that $I$ is already radical. Assume $I$ is not prime annd pick $i_1, i_2 \in I$ such that $i_1 \cdot i_2 \in I$ but $i_1 \not\in I$ and $i_2 \not\in I$. 
			Let $I_1 = \text{ ideal generated by } I \cup \{i_1\} $ and $I_2 = \text{ ideal generated by } I \cup \{ i_2\} $. 
			Then 
			\begin{enumerate}
				\item $V(I)  = V(I_1) \cup V(I_2)$

					$\supset$ : obious

					$\subset $ : Let $\mathcal{P} $ be a prime ideal containding $I$ but not $i_1$. Since $i_1i_2 \in I \subset \mathcal{P} $ we see that $i_2 \in \mathcal{P} $ and $I_2 \subset  \mathcal{P}$.

				\item $V(I) \ne V(I_1)$ and $V(I) \ne V(I_2)$
					\[
						i_1 \not\in  I \ne \sqrt{I} \text{ and } i_2 \not\in  I \ne \sqrt{I_2} 
					.\]  
					this contradicts the irreducibility of $V(I)$. 
			\end{enumerate}
	\end{enumerate}
\end{proof}
\begin{example}
	\begin{enumerate}
		\item Let $k$ be a field $R = k[X]$. 
			Then the rime deals :  $(0)$ is the unique generic point of $\spec R$ and $(p(x))$ where $p$ is monic irreducible are the closed points of $\spec R$. 


			For every $f \in k[X] \setminus \{ 0\} $ \[
				v(f) = \{(p(x)) \st p(x) \text{ monic irreducible factor of $f$}\} 
			.\] 
			\[
				V(0) = \spec R
			.\] 
		Thus the closed sets of $\spec R$ are $\phi, \spec R, \text{ finite sets of closed points} $.
		The irreducible closed subsets are $\spec R$, singletons $\{(p(x))\} $ consisting of a closed point.
		
		If  $k$ is algebraicly closed 
		\[
		\begin{tikzcd}
			k \arrow[r, leftrightarrow] & \mspec k[x] \\
			a \arrow[r, leftrightarrow] & x - a
		\end{tikzcd}
		.\] 
		under this identification, $V(f)$ is just the set of zeros of $f$ in $k$ 

		If $k$ is not algebraically closed. 
		Fix and algebraic closure $k^{a}$ of $k$. 
		\begin{align*}
			\frac{k^{a}}{G(k^{a} / k)} &\overset\sim\to  \mspec k[x] \\
			a & \mapsto (\mathcal{P}_a(x))
		.\end{align*}
		is a bijective correspondence.


	\item Let $R =\Z$. The prime ideals $(0)$ is the generic point of $\spec \Z$, $(p)$ $p$ prime are the closed points of $\spec \Z$
		For $n \in \Z, n \ne 0$, 
		\begin{align*}
			V(n) &=  \{(p) \st p \text{ is a prime divisor of } n\}  \\
			V(0) &=  \spec \Z 
		.\end{align*}
		Thus the closed subsets of $\spec \Z$ are $\emptyset, \spec Z$, finite sets of closed points. 
		The irreducible closed subsets are $\spec Z , $ singletons $\{(p)\} $ consisting of a closed point. 
		Proving tha ta property holds at the generic point of a scheme "usually" is equivalent to proving that it holds on some non-empty open subset.
	\end{enumerate}
\end{example}
\begin{figure}[ht]
    \centering
    \incfig{spec-z}
    \caption{spec Z}
    \label{fig:spec-z}
\end{figure}

\subsection{Affine schemes} \label{sec:affine_schemes}
Let $R $ be a commutative ring. Will will upgrade $\spec R$ to a locally ringed space $(\spec R, \mathcal{O} _{\spec R})$.

\paragraph{universall property} we will later prove that our construction of $\spec R$ satisfies
\begin{enumerate}
	\item $\mathcal{O} _{\spec R}(\spec R) = R$ 
	\item Let $(Y, \mathcal{O} _Y)$ be any locally ringed space. Then \begin{align*}
			\hom_{\loc \text{Sp}}: ((Y, \mathcal{O} _Y), \spec R) &\longrightarrow  \hom_\text{Rings} (R, \mathcal{O} _Y(Y)) \\
				h &\longmapsto h^\# : \mathcal{O}_{\spec R}(\spec R) 
	.\end{align*}
	This is a bijection
\end{enumerate}

The ideal is: The stalk of $\mathcal{O} _{\spec R}$ at $\mathcal{P}  \in \spec R$ should be $R_{\mathcal{P} }$. 
Now we mimick the construction of sheafification. 
For every $U \open \spec R$, let $\mathcal{O} _{\spec R}\left( U \right) $ be the set of tuples $(r_p) \in \prod_{\mathcal{P}  \in U}R_\mathcal{P} $ satisfying:
$\forall Q \\in  U$ then $\exists s \in R$ and $r \in R[\frac{1}{s}]$ with these properties:
\begin{enumerate}
	\item $Q \in D(s) \subset U$ 
	\item $\forall \mathcal{P}  \in D(s)$, $r_p$ is the image of $r$ under the localization morphism $R[1 / s] \to R_{\mathcal{P} }$ + obvious restriction maps
\end{enumerate}
\begin{exercise}
	$\mathcal{O} _{\spec R}$ is a sheaf of rings over $R$. The stalk of $\mathcal{O} _{\spec R}$ at any $\mathcal{P}  \in \spec R$ is $R_\mathcal{P} $. 
	Thus $(\spec R, \mathcal{O} _{\spec R})$ is a locally ringed space ("affine scheme")
\end{exercise}
\begin{example}
	$\spec \Z$
	The local rings will be $\Z_{(2)}, \Z_{(3)}, \Z_{(5)}, \ldots, \Q$. 
	The residue fields are $\F_2, \F_3, \F_5, \ldots, \Q$.
\end{example}
\begin{note}
	We have a natural ring morphism 
	\begin{align*}
		R & \to \mathcal{O} _{\spec R}(\spec R)\\
		r & \mapsto \text{constant tuple }(r)_{\mathcal{P} \in \spec R} 
	.\end{align*}
	Thus we can n evaluate elemetns $r \in R$ at points $\mathcal{P}  \in \spec R$, 
	\[
		r \in R \to \mathcal{O} _{\spec R}(\spec R) \to \mathcal{O} _{\spec R, \mathcal{P} } = R_{\mathcal{P} } \to \text{residue field of } \mathcal{O} _{\spec R, \mathcal{P} } = \text{Frac}(R / \mathcal{P} )
	.\] 
	This gives geometric interpretations for $V(I) , D(s)$ when $I $ is an ideal of $R, s \in R$. 
	\[
		V(I) = \{ \mathcal{P}  \in \spec R \st r(\mathcal{P} )  = 0 \text{ for every } r \\in  I\} 
	.\] 
	\[
		D(s) = \{\mathcal{P}  \in \spec R \st s(\mathcal{P} ) \ne 0\} 
	.\] 
\end{note}
\begin{proposition}
	Let $s \in R$.
	\begin{enumerate}
		\item $(D(s), \mathcal{O} _{\spec}|_{D(s)}) \simeq \spec R[1 / s]$
		\item \[
				R\left[\frac{1}{s}\right] \to \mathcal{O} _{\spec R}(D(s))
		,\] 
		is an isomorphism . In particular $R \to \mathcal{O} _{\spec R}(\spec R)$ is an isomorphism.
	\end{enumerate}
\end{proposition}
\begin{proof}
	Let $\phi: R \to R[1 / s]$ be the localisation morphism. 
	This induces a homeomorphis. \[
		\spec \phi: \spec R\left[1 / s\right] \to D(s). 
	.\] 
	If $\mathcal{P} $ is a prime ideal in $R[\frac{1}{s}]$ and $Q = \phi^{-1}(\mathcal{P} )$, then by the universal property of localization we have a natural isomorphism $R_Q \overset\sim\to R\left[ \frac{1}{s} \right] _{\mathcal{P} }$
	This yields an isomorphism of locally ringed spaces \[
		\spec R[1 / s] \to (D(s), \mathcal{O} _{\spec R}|_{D(s)})
	.\] 


	For the second part we can use (1). So it suffices to prove the second part of the satement; replacing $R$ by $R[1 / s]$ we then also get the first part. 
	\begin{enumerate}
		\item \[
		 \begin{tikzcd}
			 R \arrow[r] \arrow[dr] & \mathcal{O} _{\spec R}(\spec R) \arrow[d] & \text{ is injective}\\
				     & \prod_{\mathcal{P}  \in \spec R} R_{\mathcal{P} }
		 \end{tikzcd}
		.\] 
		if $x \in R \setminus \{0\} $ then $\text{Ann}_R(x) \ne R$, thus $\text{Ann}_R(x) \subset  \mathcal{P} $ for some prime ideal
	\item \emph{surjectivity}
		Take $(r_\mathcal{P} ) _{\mathcal{P}  \in \spec R} \in \mathcal{O} _{\spec R}(\spec R)$. We ant $r in R$ such that $r = r_{\mathcal{P} } \in R_{\mathcal{P} }$ for every $\mathcal{P}  \in \spec R$. 
		By definition of $\mathcal{O} _{\spec R}$ and quasicompactness of spec $R$, $\exists s_1, \ldots, s_n \in R$ such that \[
			\spec R = D(s_1) \cup  \ldots \cup D(s_n)
		.\] 
		and such that for every $i \in \{1, \ldots, n\} $, there exists $r_i \in R[1 / s_i]$ such that  $r_i = r_p$ in $R_\mathcal{P} $ for all $\mathcal{P}  \in D(s_i)$. 
		We want to glue these elements $r_i$ to a single element $r \in R$. 
		Aplpying (a) to the ring $R[\frac{1}{s_i s_j}]$ we see that $r_i = rj$ in $R[\frac{1}{s_i s_j}]$ because $r_i = r_j = r _{\mathcal{P} } $ for all $\mathcal{P}  \in D(s_i s_j) = \spec R[\frac{1}{s_i s_j}]$. 
		Thus $\exists m \ge 1$ such that $s_i^{m}r_i \in R$ and $s_i^{m}s_j^{m}(r_i -r_j) = 0 $ in $R$ for all $i$ and $j$.
		Replacing $s_i$ by $s_i^{m}$ for all $i$, we maay ssume $m = 1$ (this does not affect $D(s_i)$ or $R[1 / s_i]$). 
		Since  $\spec R = D(s_1) \cup \ldots\cup \spec D(s_n)$, we can find $\lambda_1, \ldots, \lambda_n$ in $R$ such that $\lambda_1 s_1 + \ldots + \lambda_n s_n = 1$. 
		(think of this a "partition of unity"). 
		Now set $r = \lambda_1 s_1 r_1 + \ldots + \lambda_n s_n r_n$. 
		In $R[1 / s_i]$ we have  $s_j r_j = s_j r_i$. 
	For all $j \in \{1, \ldots, n\} $, thus $r = \underbrace{\left( \sum_{j = 1}^{n }\lambda_j s_j \right)}_{ = 1} r_i$ in $R[1 / s_i]$. 
	The the image of $r$ in $R_{\mathcal{P} }$ equals the image of $r_i$ in $R_\mathcal{P} $ for all $\mathcal{P}  \in D(s_i) = \spec R[1 / s_i]$
	this is precisely the coordinate $r_\mathcal{P} $ fo our initial tuple
	\end{enumerate}
\end{proof}
\begin{exercise}
	Prove that $\spec R$ is disconnected if and only if $R$ contains a nontirvial idempotend element $e$, ($e^2 = e, e \in \{0, 1\} $.
	Challenge: Prove this without the previous proposition.
\end{exercise}
