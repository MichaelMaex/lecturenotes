\lecture{3}{2020-02-26}{Belangrijkste resultaten}
\section{Berekenen van Integralen} \label{sec:berekenen_van_integralen}
We willen de volgende integraal uitrekenen.
\[
\int_{-\infty}^{\infty} \frac{\sin x}{x} \dd x = \lim_{R \to \infty} \int_{-R}^{R}\frac{\sin x}{x} \dd x
.\]
Met een primitieve kunnen we dit niet uitrekenen. Deze integraal is ook niet absoluut convergent.
Beschouw de functie \[
	f:\C\to \C: z\mapsto \frac{\sin z}{z} = \frac{e^{iz}-e^{-iz}}{z i 2}
.\] 
We weten dat  $\sin x = \im e^{ix}$. 
We beschouwen dan 
\[
\int_{-R}^{R} \frac{e^{ix}}{x} \dd x
.\] 
Nu krijgen we een probleem rond $0$. Maar we kunnen met limieten werken.
Zij $0 < \epsilon < R$. 
 \[
\int_{-R}^{-\epsilon} \frac{e^{ix}}{x }\dd x + \int_{\epsilon}^{R}\frac{e^{ix}}{x}\dd x
.\] 

\begin{figure}[ht]
    \centering
    \incfig{contour-voor-sin-x-over-x}
    \caption{contour voor sin x over x}
    \label{fig:contour-voor-sin-x-over-x}
\end{figure}
Dan beschouwen we over de toy contour \[
	\lim_{\substack{R \to \infty \\ \epsilon \to 0}} 
	\int_{-R}^{-\epsilon} \frac{e^{iz}}{z} \dd z + \int_{\epsilon}^{R} \frac{e^{iz}}{z} \dd z
.\] 
Dus is \[
	\im \left( \lim_{\substack{R \to \infty \\ \epsilon \to 0}} (I_1 + I_3) \right)   = \int_{-\infty}^{\infty} \frac{\sin x}{x} \dd x
.\] 
Nu beweren we dat $\lim_{R \to \infty} I_4 = 0$. 
ML afschatting gaat net niet werken. Maar we kunnen parametriseren. \[
I_4 = \int_{0}^{\pi }\frac{e^{iRe^{it}}}{Re^{it}}R i e^{it} \dd t = i \int_0^{\pi }e^{iRe^{it}} \dd t
.\] 
Dus 
\begin{align*}
	\abs{I_4} &\le \int_0^{\pi} \abs{e^{iRe^{it}}} \dd t \\
	= \int_0^{\pi} e^{-R \sin t} \dd t 
.\end{align*}
Dit wordt duidelijk klein.
Hieruit vinden we dat \[
	\lim_{\substack{\epsilon \to 0 \\ R \to \infty}} I_1 + I_3= - \lim_{\epsilon \to 0} I_2
.\] 
Dus wordt het 
\begin{align*}
	\int_{C_\epsilon^+} \frac{1}{z} \dd z &= \int_0^{\pi}\frac{1}{\epsilon e^{it}}\epsilon i e^{it} \\
	&= \pi i
.\end{align*}
Ook is \[
\int_{C_\epsilon^{+}} \frac{e^{iz}-1}{z} \dd z \to 0 \text{ als } \epsilon \to 0
.\] 
Dus is \[
\lim_{\epsilon \to 0} - I_2 = \pi i
.\] 
Dus \[
\int_{-\infty}^{\infty} \frac{\sin x}{x} \dd x = \pi
.\] 

\section{Formule van Cauchy} \label{sec:formule_van_cauchy}
\begin{stelling}
	Zij $f$ holomorph in $\Omega$. Zij $D$ een schijf met $\overline{D} = D \cup C \subset \Omega$. 
Dan $f(z)  = \frac{1}{2\pi i} \oint_{C} \frac{f(s)}{s - z}$ voor $z \in D$.
\end{stelling}
\begin{gevolg}
	 Met inductie bekomen we dat voor elke $n \in \N$ \[
		 f^{(n)}(z) = \frac{n!}{2\pi i} \oint \frac{f(s)}{(s - z)^{n + 1}} \dd s
	 .\] 
	 Dus $f$ is willekeurig vaak afleidbaar!
\end{gevolg}
We gaan het uit een ander perspectief bekijken.
Beschouw \[
	f(z) = \frac{1}{2 \pi i} \oint_C \frac{f(s)}{s - z} \dd s 
.\] 
Bekijk $f(z)$ ronnd $z_0 \in D$. Kier $z \in D$ met $\abs{z - z_0} < \frac{1}{ 2}\min_{s \in C}\abs{s - z_0}$.
Dan 
 \begin{align*}
	 \frac{1}{s-z} &=  \frac{1}{s - z_0 -(z- z_0)} \\
	 &= \frac{1}{s - z_0}\frac{1}{1 - \frac{z - z_0}{s - z_0}} \\
	 &= \sum_{k = 0}^{ \infty} \frac{(z - z_0)^{k}}{(s - z_0)^{k -1}} \\
.\end{align*}
Er volgt (het convergeert uniform in een omgeving van $z_0$)
\begin{align*}
	f(z)  &= \frac{1}{2 \pi i} \oint_C f(s) \sum_{k = 0}^{\infty}\frac{(z - z_0)^{k}}{(s - z_0)^{k + 1}} \dd s \\
	 &= \frac{1}{2 \pi i} \sum_{k = 0}^{\infty}\left( \oint_C \frac{f(s)}{(s - z_0)^{k+1}} \dd s \right) (z - z_0)^{k}
.\end{align*}
Dus $f(z) $ is een machreeks in een omgeving van  $z_0$.
\begin{gevolg}
	[Cauchy afschatting]
	Als $f$ holomorf is in $\Omega$ en $z_0 \in \Omega$ en $\overline{D_r(z_0)}\subset \Omega$.
	Dan is \[
		\abs{f^{(n)}(z_0)}  \le \frac{n!}{r^{n}} \max_{s \in C_r(z_0)}\abs{f(s)}
	.\] 
\end{gevolg}
\begin{proof}
	We weten dat \[
		f^{(n)}(z_0) = \frac{n!}{2\pi} \oint_{C_r(z_0)}\frac{f(s)}{(s - z_0)^{n + 1}} \dd s
	.\] 
	We passen hier ML afschatting op toe. 
	 \begin{align*}
		 \abs{f^{(n)}(z_0)} &\le \frac{n!}{2\pi }2 \pi r \max_{s \in C_r(z_0)} \abs{\frac{f(s)}{(s - z_0)^{n + 1}}} \\&= \frac{n!}{r^{n}} \max _{s \in C_r(z_0)} \abs{f(s)} 
	.\end{align*}
\end{proof}
\begin{stelling}
	[Stelling Liouville]
	Zij $f$ holomorf of $\C$ (gehele functie). Als $f$ begrensd is, dan is $f$ constant.
\end{stelling}
\begin{proof}
	We schrijven $f(z) = \sum_{n = 0}^{\infty}a_nz^{n},$ met $a_n = \frac{f^{(n)}(0)}{n!}$.
	Voor elke $r \ge 0 $ is 
	\begin{align*}
		\abs{f^{(n)}} &\le \frac{n!}{r^{n}} \cdot  \max_{s \in C_r(0)} \abs{f(s)} \\
			      &\le \frac{n!}{r^{n}} \cdot \sup_{s \in \C}\abs{f(s)}
	.\end{align*}
	Dus 
	\[
		\abs{a_n} \le \frac{1}{r^{n}}\cdot \sup_{s \in \C} \abs{f(s)} \text{ voor elke } r > 0
	.\] 
	Dus voor alle $n\le 1$ geldt $a_n = 0$.
\end{proof}
\begin{stelling}
	[Hoofdstelling van de Algebra]	
Als $P$ een veelterm is van graad $n \ge 1$. Dan heeft  $P$ een nulpunt in $\C$.
\end{stelling}
\begin{proof}
	Stel $p$ heeft geen nulpunten. Dan is $z\mapsto f(z) = \frac{1}{p(z)}$ een gehele functie. Als $\abs{z} \to \infty$ dan zal $\abs{f(z)} \to 0$.
	Dan is $f$ begrensd op $\C$.
	Volgens Liouville is $f$ constant. \contra
\end{proof}

