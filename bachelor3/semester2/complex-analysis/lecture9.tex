\lecture{9}{2020-04-21}{Conforme afbeeldingen}

\chapter{Conforme Afbeeldingen} \label{chap:conforme_afbeeldingen}

\begin{definitie}
Zij $U, V$ open in $\C$. 
	Een functie $U\to V$ is een conforme afbeelding van  $U$ naar $V$ als $f$ holomorf en bijectief is. 

	$U$ en $V$ zijn conform equivalent als er een conforme afbeelding van $U$ naar $V $ bestaat. 
\end{definitie}

\begin{opmerking}
	Het volgt uit de openmapping theorem dat $f^{-1}:V \to U$ ook holomorf is.
\end{opmerking}

Het is een nodige voorwaarde dat als $f$ conform is, dat de afgeleide nergens nul zal worden. 
Dit volgt uit het argument principe door het beeld van een cirkel rond $z_0$ te beschouwen.  

\begin{propositie}
	Als $f$ holomorf is $z_0 \in U$, $f'(z_0) \ne 0$. Dan bewaard $f$ de hoeken tussen twee gladde krommenn die elkaar snijden in $z_0$. 
\end{propositie}
\begin{proof}
	Beschouw de twee krommen in $U$:  
	\begin{align*}
		\alpha: [0,1] \to U: \alpha(0) = z_0, \alpha'(z_0) \ne 0\\
		\beta[0,1] \to U: \beta(0) = z_0, \beta'(z_0) \ne 0\\
	.\end{align*}

	We weten dat $\theta = \arg \beta'(0) - \arg \alpha'(0)$. 
	\[
		\frac{(f\circ \beta)'\left( 0 \right) }{(f \circ \alpha)'(0)} = \frac{f'(z_0)\beta'(0)}{f'(z_0) \alpha'(0)} = \frac{\beta'(0)}{\alpha'(0)}
	.\]  
\end{proof}

Wat als $f'(z_0) = 0$ en $f''(z_0) \ne 0$? Dan worden hoeken verdubbeld. 
Walt als $f'(z_0) = \ldots = f^{(n-1)}(z_0) = 0$ en $f^{(n)} \ne 0$. Dan worden nhoeken met factor $n$ vermenigdvuldigd. 
Dit volgt uit lokale vorm van $f$. Schrijf  in een omgeving $D_z(z_0)$ van $z_0$: \[
	f(z) = f(z_0) + g(z)^{n} \text{ met } g \text{ holomorf } g(z_0) = 0, g'(z_0) \ne 0  
.\]
Dit kan zo want 
$f(z) = f(z_0) + (z-z_0)^{n}h(z)$ met $h(z_0) \ne 0 $. 
Er is een $r > 0$ en een funcntie $k(z)$ zodat $h(z) = k(z)^{n}$ voor $z \in D$. Neem dan $g(z) = (z - z_0)k(z_0)$. 

\begin{vb}
	$f(z) = e^{z}$ it niet conform van $\C \to \C$ want het is niet injectief. Hett is ook niet surjectief want $0 \not\in f(\C)$.
	Als we ons beperken tot $U = \{z \st 0 < \im z < 1\} $ dan $f_U$ een conforme afbeelding van $U$ nnaar $f(U)$. 
\end{vb}
\section{Mubius transformaties} \label{sec:mubius_transformaties}
Deze worden ook wel fractional linear tranformations genoemd.
\begin{definitie}
	Een \emph{Mobius transformatie} is een functie van de vorm 
	\[
		f(z) = \frac{az + b}{cz + d} \text{ met } ad - bc \ne 0
	.\] 
\end{definitie}
\begin{opmerking}
	Er zijn 7 belangrijke eigenschappen.
	\begin{enumerate}
		\item Deze fucties zijn uitbreidbaar nnaar $\overline{\C} = \C \cup \{\infty\} $ door 
			\begin{align*}
				&\begin{cases}
					f(\infty) = \frac{a}{c} \\
					f\left(-\frac{d}{c}\right) = \infty
				\end{cases} &\text{ als } c\ne 0 
				\\
				&f(\infty) = \infty &\text{ als } c = 0
			.\end{align*}
		\item $f: \overline{\C} \to \overline{\C}$ is een bijectie en de inverse functie is ook een mobius transformatie.
		\item De samenstelling van twee mobius transformaties is een mobius transformatie.
			De mobius transformaties vormen dus een groep.
		\item Er zijn 4 eenvoudige vormen 
			\begin{description}
				\item[translatie] $z\mapsto z + b$ 
				\item [rotatie] $z\mapsto e^{i\theta} z$ 
				\item [dilatatie] $z\mapsto rz$, $r \in \R$ 
				\item[inversie] $z \mapsto z^{-1}$
			\end{description}
		\item Deze vormen brengen de groep voort. 
			\begin{proof}
				\begin{align*}
					\frac{az + b}{cz + d} - \frac{a}{c} &= \frac{c(az+ b) - a(cz+ d)}{c(cz + d)} \\
									    &= \frac{bc - ad}{c(cz + d} \\
					\frac{az + b}{cz + d} &= \frac{bc - ad}{c(cz + d)} + \frac{a}{c}
				.\end{align*}
			\end{proof}
		\item Een MT wordt bepaald door 3 punten en hun beelden.
			Als  $z_0, z_1, z_2 \in \overline{\C}$ onderling verschillend en $w_0, w_1, w_2 \in \overline{C}$ onderling verschillend. Dan is er een unieke MT $f$ met $f(z_j) = w_j$.

			Neem $w_0 = 0, w_1 = 1, w_2 = \infty$. 
			Dan voldoet \[
			f: z\mapsto \frac{z - z_0}{z - z_2} - \frac{z_1 - z_2}{z_1-z_0}
			.\]  
			Analoog bestaat er een $g: w_0\mapsto 0, w_1\mapsto 1, w_2\mapsto  \infty$. 
			Dan voldoet $g^{-1} \circ f$.
		\item Een mobius transformatie beeldt een cirkel of een rechte op een cirkel of rechte.
			\begin{opmerking}
				Een rechte is eigenlijk een cirkel met oneindige straal die door de pool gaat. 
			\end{opmerking}
			Het volstaat om het te checken door de generators van de groep. 
			Het is oke voor translatie rotatie en dilatie. Is het ook waar voor inverse?
			Zij $\abs{z - a} = r \ge 0$ een cirkel. Dan kunnen we die schrijven als
			$(z - a) (\overline{z} - \overline{a}) = r^2$. 
			Neem $z = \frac{1}{w}$. 
			Dan $(\frac{1}{w} - a) (\frac{1}{\overline{w}} - \overline{a}) = r^2$. 
			Dus
			\begin{align*}
				(1 - aw) (1- \overline{a}\overline{w}) &= r^2 \overline{w}w \\
				(\abs{a}^2 - r^2) \overline{w}w - aw - \overline{aw} + 1 &= 0 
			.\end{align*}
			Als $|a|^2 - r^2  = 0$. Dan staat er de vergelijking van een rechte. 
			Neem aann dat $\abs a ^2 - r^2 \ne 0$. 
			\begin{align*}
				w \overline{w} - \frac{a}{|a|^2 - r^2} - \frac{\overline{a}}{ |a^2| - r^2}\overline{w} + \frac{1}{|a|^2 - r^2} &= 0 \\
				w \overline{w} - b \overline{w} - \overline{b} w + \abs b ^2 &= \abs b ^2 - \frac{1}{\abs a ^2 - r^2}\\ 
				(w - b)(\overline{w} - \overline{b})&=  \abs b ^2 - \frac{1}{|a|^2 - r^2}\\
			.\end{align*}	
			Dit is een cirkel als $|b|^2 - \frac{1}{|a|^2 - r^2} > 0$. 
			Maar dat is zeker zo aangezien er punten opliggen.
	\end{enumerate}

\end{opmerking}

	Maak een connforme afbeelding van $U$ naar $V$ voor zekere $U$ en $V$. 
	Met $V = \mathbb D = \{\abs z < 1\} $ 
	Dit kan als $U$ enkelvoudig samenhangend is.  (dit is de Riemann afbeelding stelling). 
\begin{vb}

	In het geval dat  $U = \C^{+} = \{\im z > h\} = \mathbb H$. 
	We kunnen een transformatie zoeken die de $\R$-as afbeeld op de eenheids cickel. Dan wordt het bovenste halfvlak op de binnen of buitenkant van de circkel afgebeeld. 

	We nemen 3 punten $0, 1, \infty$. Dus we sturen $0 \mapsto  1, 1\mapsto i, \infty \mapsto  -1$. Deze tranformatie beeld de $\R$-as af op de eenheids cirkel. 
	Het bovengebied gaat naar binnen de cirkel omdat de transformatie hoeken bewaart. 
	Vergelijk dan de normaal van de rechte en het beeld. 
\end{vb}
\begin{vb}
We nemen $U = \{z \st 0 < \arg z < \frac{\pi}{4}\} $. Hoe mappen we dit naar de eenheids cirkel?
We maken een tussenstap langs $\mathbb H$. We beginnen met  $z\mapsto z^{4}$.
\end{vb}
\begin{vb}
	Neem $U = \{0 < \im  z < 1\} $. Dit gaat niet met machten van  $z$ of de mobius transformatie. 
	De $e$ macht gaat wel. Wat is het beeld van $z\mapsto  e^{z}$. 
We weteen dat $e^{z} e ^{x + iy} = e^{x} e ^{iy}$
Dus dit wordt afgebeeld of de sector  $ \{0 < \arg e ^{z} < 1 \} $. 
We kunnen naar het halfvlak gaan door $z\mapsto  z^{\pi}$ .

We hadden dit vanaf het begin kunnen doen door eerst naar de strook $\{ 0 < \im z < \pi \} $ te gaan. En dan gaat $z\mapsto e^{z}$ naar $\mathbb{H}$.
\end{vb}
