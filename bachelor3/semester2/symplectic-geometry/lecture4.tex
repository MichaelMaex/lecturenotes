\lecture{4}{2020-03-11}{Submanifolds and Normal Forms}

\begin{revision}
	\begin{itemize}
		\item A time-dependent vector field is a fiamily $\{x_t\} _{t \in I}$ of vector fields on  $M$ which depends smoothly on $t$. 
		\item The flow of a  time dependent vector field satisfies the differential equation \[
				\frac{\dd f(\psi'_t(\rho)}{\dd t}|_{t' = t} = X_{t}(\psi_t(\rho))(f), \forall  f \in C^{\infty}(\mu), t \in (-\epsilon, \epsilon)
		.\] 
		$\psi_t: U \subset M \to U$ is an isotopy of diffeiomorphisms.
	\item Time-dep vfs on $M$ are in 1-1 correspondence with (time independent, honest) vf on $M \times (-\epsilon, \epsilon)$.
		And  $(X(t, p)$
	\item If  $\{\alpha_t\} _{t 
		\in I} \subset \Omega^{k}(M) $ smooth family, $\rho_t:M\to M$ isotopy, 
		\[
			\frac{\dd}{\dd t}(\rho^*_t(\alpha_t)) = \rho^*(\mathcal{L} _{X_t} \alpha_t + \frac{\dd }{\dd t})
		.\] 
	\item A \emph{Chain homotopy} operator betweeen smooth maps $f_0, f_1: M \to N$ is a linear map \begin{align*}
			Q: \Omega^{k}(N) &\longrightarrow \Omega^{k-1}(M) \\
		f^*_1 - f^*_0 &= \dd \circ Q + Q \circ \dd
	.\end{align*}
	\end{itemize}
\end{revision}

\begin{notation}
	\begin{align*}
		H&: I\times M \to N \text{ smooth}\\
		f_t&: M \to N, x\mapsto  H(t, x) \\
		f_0&: x\mapsto H(0, x) \\
		f_1&: x\mapsto H(1, x)\\
		I_t: M \mapsto  I\times M, x\mapsto (t, X) 
	.\end{align*}
	Then \[
	Q := \int_0^{1}I_t^*\circ i_{\frac{\partial }{\partial t} }\circ H^* \dd t
.\] is a homotyp between $f^*_0$ and $f^*_1$.
\end{notation}

The aim of the following lectures is 
\begin{itemize}
	\item  Mores theorem
	\item Darbous theorem
		\time Weinstein Lagrangiand manifold theorem.
\end{itemize}	

\begin{proposition}
	[Tuburlar neighborhood theorem]	
	Let $M$ be a smooth manifold, $i: X \into M$ (embedded) submanifold, compact, $\forall  \rpho \in M \exists $ a neighborhood.\[
	O_p \subset V \subset NX = \frac{TM|_x}{TX}
	\] 
	of the zerosection such that $\forall p \in X: V\cap N_pX$ is convex and $\exists $ diffeomorphism $\phi:V \to \phi(v)$
	such that \[
	\begin{tikzcd}
		V \arrow[r, "\phi"] & M^{P} \\
		X \arrow[u, hookrightarrow, "o"] \arrow[ur, hookrightarrow,"i"]
	\end{tikzcd}
	.\] 
\end{proposition}
\begin{proof}
	See Comprehensive introduction to differential geometry by Spivak.
\end{proof}
\subsection{Small excursion into Riemannian Geometry} \label{sec:small_excursion_into_riemannian_geometry}
\begin{definition}
	A \emph{Riemann metric} $g$ on $M$ is a fiberwise metric for $TM$, i.e. a inner product (pos-def) on every $T_pM$, st. $g_p$ varies smoothly.	
\end{definition}
\begin{note}
	This a not a metric, like a metric space, But it does induce a metric on the Riemannian manifold.
	\[
		d_g(p, q) = \inf \left\{\int_0^{1}\sqrt{g(\dot\gamma(t), \dot\gamma(t))}\dd t \right\} 
	.\] 

	The energy functional on paths is $E(\gamma) = \frac{1}{2} \int_0^{1}g(\dot\gamma(t), \dot\gammat(t)) \dd t$. 
	Geodesics minimize this functional.
\end{note}
\begin{theorem}
	Let $p \in M$. THere exists a neighborhood $U \subset M$ of  $p$ and $\epsilon > 0$ such that  \[
		\forall q \in U, V \in T_qM \text{ such that }\|v\|_g = \sqrt{g(v, v)}  \le \eppsilon:
	.\] 
	THere is a unique geodesic $\gamma_V:(-1, 1) \to M, \gamma V(0) = q, \dot \gamma_v(0) = V$.
\end{theorem}
\begin{definition}
	For any $p \in M$, the \emph{exponential map at $p$ } is \[
		\exp_p: V \subset T_pM \to M: V\mapsto \exp_p(v) = \gamma(1)
	,\]
	where $\gamma:[0,1] \to M$ is the unique geodesic such that $\gamma(0) = p ,\dot\gamma(U) = V$. 
\end{definition}

We will now prove the tublar neigborhood theorem.
\begin{proof}
	[proof of tubular neigborhood theorem]
Choose a Riemann metric $g, dg, \|\bullet\|_g$ for norm induced by $g$.
Let $\varepsilon = \{v \in  T_p M \st p \in X, v \in (T_pX)^{\perp}\} \simeq NX = \frac{TM|_X}{TX} $
$U_\epsilon = \{q \in M \st dg(q, X) \le \epsilon\} $.
So $\exp: \varepsilon_\epsilon  \to U_\epsilon$ is defined if $\epsilon$ is small enough. 
\begin{lemma}
	For sufficiently small $\epsilon$, $\exp$ is a difeomorhpism
\end{lemma}
\begin{proof}
	$V \subset E$ for set of non-critical points which contains the $0$-section.
	Write $V_1 = \overline{V\cap E_1}$ is comppact set that contains the $0$-section.
\end{proof}
\begin{lemma}
	If $A$ is a compact metric space $X_0 \subset A$ is closed. Let $f: A \to B$ be a local homeomorphism such that $f|_x $ bijective, then $f$ is bijective on a neighborhood of $X$.
\end{lemma}
If $\epsilon$ is sufficiently small then $\exp: E_\epsilon \to \exp(E_\epsilon) $ is diffeomorphism. 
Let ot show $\exp(E_\epsilon) - U_\epsilon$. 
It is clear that $\exp(E_\epsilon) \subset U_\epsilon$. 
The other inclusion is not trivial. 
Pick $\gamma:[0,1] \to M$ geodesic of length $\le \epsilon, \gamma(0) = p, \gamma(1) = q$. 
So $g(\dot\gamma\left( 0 \right) , T_pX) = 0$. 
\end{proof}


\begin{proposition}
	Let $i; X \into M$ embedded submanifold with. $U \subset M$ a tubular neighborhood ($\pi: U \to X$).
	Then 
	\begin{enumerate}
		\item $i^*: H^{\bullet}(U) \to H^{\bullet}(X)$, is an isomorphism.
		\item If $\beta \in \Omega^{k}(U)$, $d\beta = 0$ and $i^*\beta  = 0$ then $\exists \xi \in \Omega^{k-1}(U)$ st $d\xi = \beta, \xi|_X = 0$.
	\end{enumerate}
\end{proposition}
\begin{proof}
	\begin{enumerate}
		\item $\pi: U \to X$ Homotopy equivalent $\pi \circ i = \id _X$, $i\circ \pi \simeq \id_y$, 
			Where $H:I \times U \to U, (t, v)\mapsto  (tv)$ is the homotopy.
		\item $H: I \times U \to U$ induces chain homotopy $o_p$. \[
				Q: \Omega^{\bullet}(U) \to \Omega^{\bullet -1}(U)
		\]
		from $\id^*_U$ and $(i\circ \pi)^*$.
		Assume $\beta \in \Omega^{k}_{ce}(U)$ such that $i^*\beta = 0$. 
		Then 
		\begin{align*}
			(i\circ \pi)^*(\beta ' - \id_U^*\beta)) &=  d Q(\beta) - Q(d\beta) \\
			\beta &=  dQ (\beta) - 0
		.\end{align*}
		\begin{align*}
			Q(\beta) &=  \int_0^{1}\left( I_t^* \circ i_{\frac{\partial }{\partial t} } \right)  \circ H^*(\beta)  \dd t\\
				 &= \int_0^{1} f^*_t(i_v\beta) \dd t 
		,\end{align*}
		where $V := H_*(\frac{\partial }{\partial t} ), f_t: H|_{\{t\} \times i(X)}$

	\end{enumerate}
\end{proof}

\begin{TODO}
	All geodesics were denoted wrong and should be defined on an interval $[0, \epsilon']$ with  $\epsilon'< \epsilon $.
\end{TODO}
\section{Darboux' theorem} \label{sec:darboux'_theorem}
\begin{theorem}
	[Moser's]
	Let $M$ compact, $\{w_t\} _{t \in I}$ smooth family of symp forms such that $[w_t] \in H^2(\mu)$ const, then there exists isopy $\{\rho_t\} _{t \in T}: M \to M$ such that $\rho_t^*w_t = w_0$.	
\end{theorem}
\begin{proof}
	Recall; on compact manifolds 
	\[
		\{\text{isotopies}\}  \overset{1-1}{\longleftrightarrow} \{\text{time-dep vfs}\}	
	.\] 
	Then 
	\begin{align*}
		\frac{\dd }{\dd t} \rho^*-t w_t = \pho^*_t\left( \mathcal{L} _{V_t}w_t + \frac{\dd}{ \dd t}w_t \right) 
	.\end{align*}
	where $\rho_t$ is flow of $V_t$. Need ot find $V_t$ such that $\mathcal{L} _{V_t}w_t + \frac{\dd}{ \dd t}w_t = 0$. 
	Note that $w_t - w_{t'}$ is always exact by assumption. 
	\begin{fact}
		$\exists \eta_t \in \Omega^1(M)$ such that $\dd \eta_t =\frac{\dd }{\dd t } w_t$ 
	\end{fact}
	$V_t$ unique time-dependent vector field such that $i_{V_t}w_t + \eta_t = 0$
Then the flow $\{\rho_t\} _{t \in I}$ of $V_t$ satisfies $\rho^*_tw_t = w_0$. 
\end{proof}
