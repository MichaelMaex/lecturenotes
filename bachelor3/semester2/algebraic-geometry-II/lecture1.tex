\lecture{1}{2020-02-13}{Introduction to Schemes}
\setcounter{chapter}{-1}
\chapter{About the Course} \label{chap:about_the_course}

This course givse and introduction to \emph{schemes}. This is a vast generalisation of quasi-projective varieties. Allows problems from ANT.

The main applications in this course are 
\begin{enumerate}
	\item Riemann-Roch \& classification of curves.
	\item Weil conjectures 
\end{enumerate}

Practical information and announcements will be on Toledo. The main reference for this course is \emph{The Rising Sea} by Ravi Vakil.
There are also lecture notes summarizing a selection of topics.

Evaluation will be like AG1. 
\begin{itemize}
	\item Two take home tests during the semester. 
	\item Take home exam: Choose one of the two forms
		\begin{itemize}
			\item list of questions
			\item write a short paper
		\end{itemize}
\end{itemize}

Best way to study for this course is to do all the exercise in the course. 
There will be five problem classes, which will be superviesed by Kien.

\chapter{Motivation for Schemes} \label{chap:motivation_for_schemes}
Why are quasi-projective varieties still not enough?
\section{Non-reduced Structure} \label{sec:reduces_structure}

		Let $k$ be an algebraically closed field of char $\ne 2$.
		 \[
			 C = \{(x,y) \in k^2 | x = y^2\} \subset \aff^2
		.\] 

We will attacht a geometric object called $\spec A$ to any commutative ring  $A$.


Condider now the intersection of  $C$ with the line $l = Z(x = 0)$.
In quasi projective varieties this intersection is  $(0,0)$. We loose information on the multiplicity. 
In schemes this becomes  $\spec \frac{k[x,y]}{(x - y^2, x)} \simeq \frac{k[y]}{(y^2)}$
schemes to keep track of multiplicity of intersections. 

This gives us an algebraic version of differential calculus. We can define things like tangentspaces, derivatives, differential forms, \ldots

\section{Arithmetic applications} \label{sec:arithmetic_applications}
Fundamental insight: There is a deep interplay between arithmetic properties of polynomials over non algebraically closed fields $k$ and the geometry of the algebraic varieties defined over an algebraically closed extension of $k$.

As an example we can consider Fermat's Last Theorem. To really understand FLT we have to understand the geometry of elliptic curves.  Another example is the Mordell conjecture, proven by Faltings (1980s).

\begin{theorem}
	[Mordell]
	Let $F(x,y,z) = a x^{k} + b y^{k} + c z^{k}$, where $a, b, c \in \Q$. 
	The set of solutions with coordinates in $\Q$ is finite if $\deg d \ge 4$ iff the genus of the curve defined by $F$ over $\C$ is $\ge 2$. 
\end{theorem}

Another example is given by the weil conjectures. It concerns systems of equations over $\F_q$. The idea is to count the solutions over  $\F_q, \F_{q^2}, \F_{q^3}, \ldots$. There is a pattern explained by the existence of a topology on varieties over $\F_q$ with properties similar to the ones over $\C$.  Consider
\[
\F_{q^2} \to \F_{q^2}: x \to x^{q}
.\] 
Then $\F_q$ is the set of fixed points. 
Likewise we find  $\F_{q^{d}}$ as a fixed locus of $\frob^{d}$ in $\F_q^{d}$ (algebraic closure of $\F_q$). 

Algebraic topology, would allow use to compute fixed points using Defschetz' trace formula, but this doesn't work for the zarisky topology. 
The solution is schemes + \'etale  topology.

This shows that we need a theory of algebraic geometry over arbitrary fields and maybe even over commutative rings. This cannot be done with quasi-projective varieties.
Suppose homogenous equation  $x^2 + y^2 + z^2 = 0$ over $\R$.
Over $\C$ the conic defined by this equation is just isomorphic to $\proj^1$.
But from a numbetheory standpoint there are no real solutions, but $\proj^1$ has $\infty$ many points. 

\paragraph{Naive attempt}
Define a projective varieties by looking at solution sets over $\R$ instead of $\C$. 
This is not satisfactory. Our equation would define the empty variety. 
\begin{example}
	Consider the product \[
		\prod_{a \in \F_q} (x - a)
	.\] 
	This defines $\F_q$ over $\F_q$. The solution set is just the affine line. 
	Over an algebraic closure the equation still defines a finite number of points, but  $\aff^{1}$ is a curve.
\end{example}
We somehow have to keep track of solutions over all possible extensions of the field $k$. 
Weil made such a formalism in 1940, but it was cumbersome and didn't work when $k$ was not a perfect field.
The theory of schemes developed by Serre, Grothendieck, \ldots in 1950-60' is the answer.

\section{Gluing algebraic varieties} \label{sec:gluing_algebraic_varieties}
A quasi projective avariety is defined by \emph{global} equations in $\proj^{n}$. 
But often we would like to construct varieties by first constructing local pieces and then gluing these together as in diff geom.
This is particularly useful in the theory of Moduli spaces. These are varieties that parameterize other varieties or algebraic objects on varieties (eg subvarieties in $\proj^{n}$).
The result if no longer a qproj varietu, but it is a scheme.
Thes space is qprojcetive but this is only clear a posterior. Finding the equation for moduli spaces is difficult.

\section{Philosophy of Schemes} \label{sec:philosophy_of_schemes}
\begin{enumerate}
	\item The main idea is to attacha geometric object $\spec A$ to any commutative ring $A$. 
Then construct general schemes by $\emph{gluing}$ pieces of the form $\spec A$. 
\item The central ideal is 
	\[
	\begin{tikzcd}
		\text{geometry of schemes} \arrow[r, leftrightarrow] & \text{algebraic properties of the functions that live on the scheme}
	\end{tikzcd}
	.\] 

\end{enumerate}
The geometry object is a locally ringed space, which is a tologycial space + sheaf of funcions. 
\begin{example}
	\begin{itemize}
		\item differentiable manifolds
		\item complex manifolds
		\item schemes
	\end{itemize}
\end{example}

\chapter{Sheaves} \label{chap:sheaves}
\section{Motivating example: differentiable manifold} \label{sec:motivating_example:_differentiable_manifold}
\begin{definition}
	An $n$-dimensional differentiable manifold is 
	\begin{itemize}
		\item a Hausdorff second countable topological space $X$. 
		\item An equivalence class of smoothly compatible atlases
	\end{itemize}
\end{definition}
\paragraph{Key idea} The differentaible structe on $X$ is completly determined by the sets of differentiable functions $f: U \to \R$ on all opens $U $ of $X$. 
From this data we can reconstruct an atlas. 
\begin{itemize}
	\item For every open  $U \subset X$, a continous map $\phi: U \to \R^{n}$ is differentiable iff for every open $V \subset \R^{n}$ and every differentiable function $g: V \to \R$ the composite function $g \circ \phi: \phi^{-1}(V) \to \R$ is differentiable. 
	\item For every open $V \in \R^{n}$, a continuous map $\psi: V \to X$ is differentaible iff for every open $U \subset X$  and every differentiable function  $f: U \to \R$ the composite $f\circ \psi: \psi^{-1}(U) \to \R$ is differentiable.
\end{itemize}
Now a homeomorphism $\phi: U \subset  X \to  V \subset  \R^{n}$ is a chart iff $\phi$ and $\phi^{-1}$ are differentiable.
The collection of all these charts is an atlas htat is equivalent to the original one.

Now we will forget about charts and atlases and axiomise a differentaible structure.

\begin{definition}
	[equivalent to differentiable manifold]
	\begin{itemize}
		\item Hausdorff secondcountable topological space $X$. 
		\item A map $\mathcal{O} _x: U \to \mathcal{O}_X(U)$, which maps an open in $X$ to a set of functions $f: U \to \R$ satisfying:
			\begin{enumerate}
				\item "differentiability is a local property"
					If $U$ is an open of $X$ and $\{U_i | i \in I\} $ is an open cover of $U$, then a function $f: U \to \R$ belongs to $\mathcal{O} _x(U)$ iff $f|_U:U_i \to \R$ belongs to $\mathcal{O} _x(U_i)$ for every $i \in I$.
				\item "locally,  $(X, \mathcal{O} _X)$ looks like an open in $\R^{n}$" 
					Every points in $X$ has an open eighbourhood $U$ with a homeomorphism $\phi: U \to V \subset \R^{n}$, with $V$ open, such that for every open $U'\subset U$, a funcniton $f: U' \to \R$ lies in $\mathcal{O} _x(U')$ iff the funciton $f \circ \phi^{-1}: \phi(U') \to \R$ is differentiable. 
			\end{enumerate}
	\end{itemize}
\end{definition}
The map $\mathcal{O} _x$ is called the \emph{structure sheaf} of $X$, and $(x, \mathcal{O} _X)$ is an example of a locally ringed space.


\begin{description}
	\item[Axiom 1] essential property of a sheaf
	\item[Axiom 2] determines the kind of geometry we are doing. 
\end{description}
