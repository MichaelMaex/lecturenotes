\lecture{7}{2020-04-27}{factorisatie algoritmes, factorbasis methodes}
\subsection{Factorbasis methodes} \label{sec:factorbasis_methodes}
Net zoals de algebraische groepsmethodes kijken we naar de ontbinding
\[
	\frac{\Z}{(n)} \simeq \F_p \times \F_q
.\] 
\begin{lemma}
	De vergelijking $x^2 = 1$ over $\frac{\Z}{(N)}$ heeft precies 4 oplossingen. 
\end{lemma}
\begin{proof}
	De vergelijking $(x_1^2, x_2^2) = (1, 1)$ over $\F_p \times  \F_q$ heeft 4 oplosingingen.
\end{proof}
De strategie is een "lukrake" oplossing in $v \in \frac{\Z}{(N)}$ van de vergelijking $x^2 = 1$. 
Stel $a = b-1$. Dan komt $a$ overeen met $(0, 0), (-2, 0), (0, -2), (-2, -2)$. Dus hebben 50 procent kans op factorisatie.

De zeer naieve methode is $b$ willekeurig kiezen. 
We milderen onze verwachtingen, maar we hopen dat $b^2$ als geheel getal $\in \{0, \ldots, N-1\} $ splitst in termen van $\mathcal{F} $ (splitst in kleine factoren).
Dan hebben we identiteit $b = p_1^{e_1}\cdots p_r^{e_r}$. 
We verzamelen veel dergelijke $b$ (laten we zeggen  $r + 1$). 
Dan hebben we 
\begin{align*}
	b_1 &= p_1^{e_{11}}p_2^{e_{21}} \cdots p_r^{e_{r1}}\\
	b_2 &= p_1^{e_{12}}p_2^{e_{22}} \cdots p_r^{e_{r 2}} \\
	    &\vdots \\
	b_{r +1} &= p_1^{e_{1, r + 1}} p_2^{e_{2, r + 1}} \cdots p_r ^{e_{r, r + p}} 
.\end{align*}
We bewereken een aantal relatie te kunnen selecteren waarvan het product van de vorm $(\alpha)^2 = p_1^{2a_1}p_2^{2a_2}\cdot p_r^{2a_r} = \beta^2$. 
Dit is lineaire algebra modulo 2. 
Dan hebben we $\left( \alpha / \beta \right)^2=1 $.

\begin{oef}
	Toon aan dat we ook $\gcd(\alpha - \beta, N)$ kunnen uitrekenen en dat dit dezelfde factor zal geven. 	
\end{oef}

\begin{opmerking}
	Wat is de rol van $B$ ($\mathcal{F}  = \{\text{priemen} \le B\} $.
	Grotere $B$ betekend 
	\begin{itemize}
		\item 
	meer kans om relaties te vinden. Iedere factorisatie duurt langer. 
\item 
	Het aantal relaties die nodig zijn om een niet-triviale combinatie te vinden groeit.
\item 
	De hoeveelheid werk om de kern te bereken groeit. 
	\end{itemize}
	Het blijkt dat de optimale  ongeveer $L_N(\frac{1}{2}; \frac{1}{2})$. Dit leid tot een algoritme met complexiteit $L_N(\frac{1}{2}, 2)$. 
	Deze methode heeft "Dixon's random squares method".
\end{opmerking}

er zijn een aantal optimalisaties.
\begin{itemize}
	\item neem $-1$ op in de factorbasis! Voordeel: ipv $b^2$ te beschouwen als als in $\{0, \ldots, N-1\} $ kunnen we nu werken met $\{-\left\lfloor \frac{N}{2}, \ldots, \left\lfloor \frac{N}{2} \right\rfloor \right\rfloor\} $.
	\item Neem $b$ van de vorm $\left\lfloor \sqrt{N}  \right\rfloor + k$ met $k$, klein, $\le \sqrt{n} / 3$. Dan is $b^2 \mod N = b^2 - N$.  
	\item "sieving" is boekhoudkunde om smooth numbers sneller te kunnen opsporen. 
		Als $b | b^2 - N$ dan geld dat ook voor $p|(b + \lambda_p)^2 - N$. 

\end{itemize}
Als we enkel de eerste twee optimalisaties toepassen vinden we een algoritme van $L_N\left(\frac{1}{2}, 1\right)$. Met de numberfield seave kunnen we $L_N\left(\frac{1}{3}; \sqrt[3]{\frac{64}{9}}\right) $.

