\lecture{6}{2020-04-05}{Handtekeningen gebaseerd op elliptische krommen}

\section{Handtekeningen op Basis van Elliptische Krommen} \label{sec:handtekeningen_op_basis_van_elliptische_krommen}
\subsection{Het ECDSA protocol} \label{sec:het_ecdsa_protocol}

Allice wil een bericht naar Bob sturen en wil die handtekenen zodat Bob zeker is dat Alice de boodschap heeft verstuurd.

We gebruiken $E(\F_p), \oplus, P \in E(\F_p), n = \# \left< P \right> = \ord\left(P \right) $ 
Een private sleutel $a \in \{1, \ldots, n-1\} $ en een publike sleutel $a P$.

Het stappenplan
 \begin{enumerate}
	 \item beschouw $M$ als element van $\frac{\Z}{(n)}$ (via hashfunctie).
	 \item Kies uniform random $\lambda \in \{1, \ldots, n - 1\} $ (copriem met $n$ )
	 \item Bereken $R = \lambda P = (r_1, r_2) = (r_1 : r_2 :1)$
	 \item Interpreteer  $r_1 \subset  \F_p$ als elemennt van  $\Z / (n)$ (door bijvoorbeeld een representant van $r_1$ mod $p$ te kiezen en dit mod $n$ te beschouwen). 
	 \item De handtekening $S$ is het koppel $(r_1, \underbrace{\lambda^{-1}(M + r_1a)}_{= s})$. 
		 Dit is de handtekening.
\end{enumerate}
De handtekening is $S = (r_1, s)$ met $\lambda P  = (r_1, r_2), s = \lambda^{-1}(M + r_1a)$ inverteerbaar  mod $n$. 
Verificatie
\begin{align*}
	\s^{-1}(r_1Q + MP) &=  s ^{-1} (r_1 a P + MP)  \\
			   &= s ^{-1} (M + r_1a)P \\
			   &= \lambda  P = R = (r_1, r_2)
.\end{align*}
Bob checkt dan of de $x$-coordinaat van $s ^{-1}(r_1 Q + MP)$ inderdaad gelijk is aan $r_1$.
Indien het gelijk is is bob overtuigd dat Alice de handtekenign heeft geplaatsd.


Sony in 2010 gebruikte ECDSA om hun software voor de PS3 te handtekenen. De hadden een fout gemaakt. Ze namen $\lambda$ niet willekeurig, maar altijd dezelfde. 
$S = \lambda ^{-1}(M + r_1a )$. Stel ze sturen nu $s' = \lambda ^{-1}(M' + r_1'a)$. 
Als we de twee berichten delen vinden we  \[
\frac{s}{s'} = \frac{M + r_1a}{M' + r_1'a}
.\] 
Dit kan opgelost worden naar $a$. 


ECDSA is een ellpitische-krommen-variant van DSA. Dit dateert uit 1991. 
Echter reeds in 1989 had Schnor een alternatief system beschreven waarvan men kan "bewijzen" dat handtekenen vervalsen equivalent is met discrete logaritmes bereken.
Schnor heeft er een patent op genomen. In 2008 is het patent gevallen. 

Heel kort het protocol
\begin{enumerate}
	\item Kies $\lambda \in  \{1, \ldots, n\} $ en bereken $R = \lambda P = (r_1, r_2)$ 
	\item Hash $M  | | r_1$ samen tot een element  $H \in \Z / (n)$. 
	\item bereken $s = \lambda - Ha$, handtekening is $(H, s)$
\end{enumerate}
Verificatie 
\begin{align*}
	s P + H &=  sP + aHP \\
		&= (s + aH) P \\
		&= \lambda p = R = (r_1, r_2)
.\end{align*}
checkt of $\text{hash}(M | | r_1) = H$. 

\section{Algoritmes voor Factorisatie} \label{sec:algoritmes_voor_factorisatie}

Zij $N = pq$ met $pq$ grote priemgetallen. 
Er zijn 3 soorten algoritmes
\begin{enumerate}
	\item Naieve methodes
		\begin{itemize}
			\item trial division \hfill $L_N(1;\frac{1}{2})$
			\item Pollard-p \hfill $L_N(1, \frac{1}{4})$
		\end{itemize}
	\item Algebraische groepsmethodes \hfill $L_n(\frac{1}{2}, \sqrt{2}) $
	\item Factorbasismethodes \hfill $L_n(\frac{1}{3}, \sqrt[3]{\frac{64}{9}}) $
\end{enumerate}
 
 \begin{definitie}
	 [L-notatie]
	 \[
		 L_N(\gamma, c) = e^{(c +  o(1)) (\log N)^{\gamma}(\log \log N)^{1- \gamma}}
	 .\] 
	 Een soort interpolatie tussenn polynomiaal en exponentieel.
\end{definitie}
\begin{opmerking}
	De invoergrootte is $\log N$. 
	\begin{align*}
		\gamma &=  1 &(\log N)^{c + o(1)} \tag{polynomiaal}\\
		\gamma &= 0 &e^{(c + o_{1}) \log N} \tag{exponentieel}\\	
	.\end{align*}
	Als $0< \gamma< 1$ dan noemen we het subexponentieel.
\end{opmerking}

\subsection{Na\"ive methodes} \label{sec:na\"ive_methodes}
\subsubsection{Trial division} \label{sec:trial_division}
We testen gewoon alle candidaat priemen of ze delers zijn van $N$. Dit is $O(\sqrt{N}) = L_N(1, \frac{1}{2})$.

\subsubsection{Pollard's $P$-methode} \label{sec:pollard's_$p$-methode}
We weten wegens crt het ringsmorphisme \[
\frac{\Z}{N\Z} \simeq \F_p \times \F_q
.\] 
We nemen een random walk in $\F_p$. Wat is een random walk? Het herhadelijk iterern van een functie $f: \F_p \to \F_p: x \mapsto x^2+1$. 
We zoeken hoe lang het duurt voor een waarde om 2 keer voor te komen. 
We verwachten dit na $\sqrt{p} $ stappen. 
We kunnen wel niet een lijst van alle waarden opslaan. 
We passen \emph{Floyd's cycle finding trick} toe. 

We leggen hetzelfde pad af maar aan dubbele snelheid. $y_0 \mapsto y_1 \mapsto \ldots.$ met $y_{i} = f(f(y_{i_1}))$. 
Dan na $O(\sqrt{q} )$ vinden we $x_i = y_i$. 


Dus beschouwe de functie \[
f: \frac{\Z}{N\Z} \to \frac{\Z}{N\Z}: x \mapsto x^2 + 1
.\]
EN $x_i = f(x_{i-1}), y_i = f^2(x_{i-1})$. 
Na $O(\sqrt{p} ) $ stappen is $x_i - y_i \equiv 0 \mod p$.

\subsection{Algebraische Groepsmethodes} \label{sec:algebraische_groepsmethodes}

Prototype is Pollards  $(p-1)$-methode.
Dus 
 \begin{align*}
	 \left( \frac{\Z}{N\Z} \right) ^{\times } \simeq \F_p^{\times } \times \F^{\times }_q
.\end{align*}
Onze hoop is een $M$ te vinden zodat $\#\F_p^{\times  } = p-1 | M$ en $\# \F_q^* = q-1 \not | M$.

Neem een lukrake $a \in \left( \frac{\Z}{N\Z} \right) ^{\times }$. Dan is 
\begin{align*}
	a^{M} \equiv 1
.\end{align*}
Hoe vinden we zo'n $ M$?
Maar als $p-1$ zou splitsen in kleine priemfactoren. $p-1 = \prod_{i = 1}^{r}l_i^{e_i}$ met $l_i \le B$ dan is dit een deler van $\prod_{l \le B}l ^{\left\lfloor \log_l N \right\rfloor}$ 

\paragraph{Concreet} Pollard's $(p-1)$-methode van smoothnes bound $B$, 
Kies $a \in (\frac{\Z}{N})^{\times }$. 
Dan $b = a$. 
Voor $f \in \{\text{priemen }\le B\} $. 
Zie cursus voor het algoritme.

De kans dat het toepasbaar is is echter zeer klein. 

\subsubsection{Pollard's $(p+1)$-Methode} \label{sec:pollard's_$(p+1)$-methode}

We werken met de groep $\F^{\times }_{p^2} / \F^{\times }_p$ met $p + 1$ elementen.
We nemen $\F_{p^2} = \frac{\F_p[x]}{x^2 -a}$ met $a \in \F_p$ geen kwadraat als model voor $\F_{p^2}.$.
Als we $M$ kunnen vinden zodat $p + 1 | M$. 

Stel we vinden een $a$ dat geen kwadraat is mod $p$ noch mod $q. $ dan is
 \[
	 \frac{\Z / N\Z [x]}{(x^2 - a)} \simeq \frac{F_q[x]}{(x^2 - a)} \times  \frac{\F_q[x]}{(x^2-a)} \simeq \F_{p^2}\times  \F_{q^2}
.\] 
Stel $p + 1 | M$ maar $q + 1 \not | M$. 
Kies $B \in \frac{\Z / N \Z [x]}{(x^2 - a)} $ lukraak.
Als $b^{M} = \alpha x + \beta$  dan is $\alpha$ deelbaar door $p$ maar niet door $q$ dus  $p = \gcd(\alpha, N)$.

\begin{opmerking}
	Hoe vinden we zo'n $a$?
	Door lukraak te proberen. De kans is 1/2 dat dit geen kwadraat is mod $p$. 
\end{opmerking}

\subsubsection{Lenstra's ECM (ellpitic curve method, 1983)} \label{sec:lenstra's_ecm_(ellpitic_curve_method,_1983)}

We beschouowen $E: y^2 = x^3 + A x + b,\; A, B \in \Z / N \Z$.
We beschouwen dit modolo $p, q$. 
Dan is op een subtiliteit of oneindig na 
\[
	E\left( \frac{\Z}{N\Z} \right) \simeq E_p(\F_p) \times E_q(\F_q) 
.\] 
Hoop: vind een $M$ zodat $\# E_p(\F_p) | M, \# E_q(\F_q) \not |M$. 
Dan kunnen we een lukraak punt  $p \in E\left( \frac{\Z}{N\Z} \right) $ nemen. Dan is $M*P = \mathcal{O}  \text{ op }E_p(\F_p) $ maar $\ne \mathcal{O}  \text{ op } E_q(\F_q)$.

Twee punten sommeren tot oneindig als ze vebonden zijn door een verticale rechte. 
Dus als $\lambda = \frac{3x_1^2 + A}{2y_1}$. 
Dus bij een bepaalde stap in de brekening $M_P$ gaan we een noemer krijgen die we niet kunnen inverteren. Dus is $\gcd(y_1, N) = p$. 

Hoe kiezen we zo'n lukraak punt? Hier is dat absoluut niet triviaal. 
Hoe vinden we zo'n oplossing
\[
y^2 = x^3 + Ax + B \text{ over } \frac{\Z}{N\Z}
.\] 
Dit gaat echter niet. Er is wel een truc om dit te omzeilen. 
Kies een punt en kies dan een $A, B$ zodat het punt op de curve ligt.

\begin{opmerking}
	Wat is de kans op succes? Zij $M = \text{product van kleine factoren}$,  dus als $\#E_p(\F_p)$ spitst in kleine factoren. 
	Dit is vergelijkbaar met de vorige methodes. Maar in dit geval kunnen we een andere kromme nemen en opnieuw proberen. 
\end{opmerking}
\begin{herinner}
	Hasse: $\#E_p(\F_p) \in [p + 1 - 2 \sqrt{p} , p + 1 + 2 \sqrt{p} ]$
\end{herinner}
Volgens Lenstr is er succes na een sub-exponentieel aantal pogingen, $L_N(\frac{1}{2})$.
