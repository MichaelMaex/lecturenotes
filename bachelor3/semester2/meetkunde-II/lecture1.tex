\lecture{1}{2020-02-10}{Inleiding en projectieve ruimtes}
\setcounter{chapter}{-1}
\chapter{Afspraken}
\begin{enumerate}
	\item examen
		\begin{itemize}
			\item enkel handgerschreven notas
			\item mondeling en schriftelijk
		\end{itemize}
	\item Cursus bestaat uit twee delen, algebraische en diff meetkunde.
\end{enumerate}

\section{overzicht}
\begin{itemize}
	\item Deel 1: Algebraische meetkunde
		\begin{itemize}
			\item Algebraische hyperoppevlakkken
			\item Krommen in het vlak, zowel affien als projectief

		\end{itemize}
\end{itemize}
\begin{vb}
	Kromme (zie \cref{fig:voorbeeld-kromme-1}): \[
	\{(x,y) \in \mathbb{A} \mid x^3 - y^2 = 0 \} 
.\] 	
\begin{figure}[ht]
    \centering
    \incfig{voorbeeld-kromme-1}
    \caption{voorbeeld kromme 1}
    \label{fig:voorbeeld-kromme-1}
\end{figure}
Oppervlak: \[
	\{(x,y,z) \in \mathbb A \mid x^2 + y^2 - z^2 + z^3 = 0\} 
.\] 

\end{vb}
\begin{itemize}
	\item Deel 2: Differentiaal Meetkunde
		\begin{itemize}
			\item We zullen oppervlakken in $E_3$  bestuderen. 
		\end{itemize}
\end{itemize}
\begin{figure}[ht]
    \centering
    \incfig{compact-oppervlak}
    \caption{compact oppervlak}
    \label{fig:compact-oppervlak}
\end{figure}

\part{Inleiding tot Algebraische Meetkunde}
\chapter{Projectieve meetkunde over willekeurig veld}
Zij $V$ een vector ruimte over een veld $K$. \[
V_0 := V \ \{0\} \;\; K_0 := K\\{0\} 
.\] 
Definieer op $V_0$ de volgende euqivalentierelatie \[
v \sim w \iff \exists \lambda \in K_0: v = \lambda w 
.\] 
We moeten checken dat dit (oefening) 
\begin{itemize}
	\item symmetrisch
	\item reflexief
	\item transitief
\end{itemize}
is. 

De equivalentieklassen vorm een partitie van $V_0$. We noteren $[v]$ voor de equivalentie klasse van $v$. 
\begin{align*}
	[v] &= \{ w \in V \mid w \sim v\}  \\
	&= \{\lambda v \mid \lambda \in K_0\} 
.\end{align*}
\begin{definitie}
De projeciteve ruimte geassocieerd aan $V$ is \[
	P(V) := V_0 / \sim = \{ [v] \mid v \in V_0\} 
.\] 	
Een element $[v]$ noemen we een \emph{projectief punt met representant $v$}.
De dimensie van $P(V)$ is $\dim P(V) = \dim V - 1$. 
\end{definitie}

\begin{vb}
	\begin{enumerate}
		\item Beschouw \[
		V = K^{n +1} = \left\{ \begin{pmatrix} x_0 \\ x_1 \\ \vdots \\ x_n \end{pmatrix} \middle | x_0, \ldots, x_n \in K \right\} 
		.\] 
		Dan geeft dit de \emph{standaard pojectieve ruimte van dimensie $n$ over $K$ }
		\[
			P(V) = \left\{ \left[ \begin{pmatrix} x_0 \\ \vdots \\ x_n \end{pmatrix}  \right] \middle | \begin{pmatrix} x_0 \\ \vdots \\ x_n \end{pmatrix} \ne 0  \right\} 
		.\] 
	\item Beshouw \[
			V = \{0\} \;\; P(V) = \emptyset
	.\] 
	Merk op dat deze ruimte dimensie -1 heeft. 
\item $\R P^{n}$ 
	Kies een hypervlak in $\R^{n+1}$, gezien als affiene ruimte $\aff^{n+1}$.
	Neem een $v \in \R^{n+1} \setminus \{ 0\} $. Dan zijn er twee mogelijkheden
	\begin{enumerate}
		\item $v \in H \implies [v] \cap (p_0 + H) = \emptyset$
		\item $v \not\in H \implies [v] \cap (p_0 + H)$ bestaat uit juist een punt.
	\end{enumerate}
	Dus we kunnen zeggen dat \[
		\R P ^{n } = \underbrace{(p_0 + H)}_\text{$n$-dimesnionale affiene ruimte} \cup \underbrace{P(H)}_\text{$n-1$ dim projectieve ruimte}
	.\] 

\item 
	$\mathbb F_2 = \{0,1\} $ veld met twee elementen. Dan is $\mathbb F_2 P^{1} = \{(1,0),(0,1),(1,1)\} $
\item Beschouw $\mathbb F_3 = \{0,1,2\} $. Dan is \[
\mathbb F_2P_1 = \left\{ \left[ \begin{pmatrix} 1 \\ 0 \end{pmatrix}  \right] , \left[ \begin{pmatrix} 1 \\ 1 \end{pmatrix}  \right], \left[ \begin{pmatrix} 0\\ 1 \end{pmatrix}  \right]  , \left[ \begin{pmatrix} 1 \\ 2 \end{pmatrix}  \right]  \right\} 
.\] 
	\end{enumerate}
\end{vb}
\begin{figure}[ht]
    \centering
    \incfig{affien-projectief}
    \caption{affien projectief}
    \label{fig:affien-projectief}
\end{figure}

\begin{definitie}
	Zij $W \subset V$ lineaire deelruimte. Dan is $P(W) \subset  P(V)$ een \emph{projectieve deelruimte}.
\end{definitie}
\begin{definitie}
	Zij $X_0 = [v_0], \ldots, X_k = [v_k] \in P(V) $ zijn \emph{projectief (on)afhankelijk} als de representanten lineair (on)afhankelijk zijn.
\end{definitie}
\begin{definitie}
	De projectieve deelruimte van $P(V)$ voortgebracht door $X_0 = [v_0], \ldots, X_k = [v_k] \in P(V) $ is $P(\left< v_0, \ldots, v_k \right>)$.
\end{definitie}
\begin{figure}[ht]
    \centering
    \incfig{rechte-voortgebracht}
    \caption{rechte voortgebracht door twee punten}
    \label{fig:rechte-voortgebracht}
\end{figure}
\begin{vb}
	Zij \[
	X_0 = \left[ \begin{pmatrix} 0 \\ 0 \\1 \end{pmatrix}  \right] , X_1 = \left[ \begin{pmatrix}  1 \\ 0 \\ 2\end{pmatrix}  \right] \in \C P^2
	.\] 
	Deze punten zijn projectief onafhankelijk en brengen een projectieve rechte $l = P\left(\left<\begin{pmatrix} 0 \\0 \\1 \end{pmatrix} , \begin{pmatrix} 1 \\ 0 \\2 \end{pmatrix}  \right>\right)$ voort. 
	Dus \[
	 l = \left\{ \left[ \begin{pmatrix} x_0 \\ x_1 \\ x_2\end{pmatrix}  \right] \in \C P^2 \middle | x_1 = 0  \right\}
	.\] 
	\[
	l \leftrightarrow \begin{pmatrix} x_0 \\ x_1 \\ x_2 \end{pmatrix} = \lambda_0 \begin{pmatrix} 0 \\0\\1 \end{pmatrix}  + \lambda_1 \begin{pmatrix}  1 \\ 0 \\ 2 \end{pmatrix}  
	.\] 
\end{vb}
\begin{stelling}[dimentiestelling]
	Zij $P_1, P_2 \subset P(V)$ projectieve deelruimten van een eindige dimensionale projectieve ruimte. Dan is $\dim P_1 + \dim P_2 = \dim (P_1 \cap P_2) + \dim (P_1 + P_2)$.
\end{stelling}
\begin{proof}
	Stel $P_1 = P(W_1), P_2 = P(W_2)$. Dan zegt de dimentiestelling voor lineaire deelruimten dan 
	\begin{align*}
		\dim W_1 + \dim W_2 &= \dim (W_1 \cap  W_2) + \dim (W_1 + W_2) \\
		\dim P_1 + \dim P_2 &=  \dim P(W_1 \cap  W_2) + \dim P(W_1 + W_2) \\
		\dim P_1 + \dim P_2 &=  \dim (P(W_1) \cap P(W_2) ) + \dim (P(W_1) + P(W_2)) \\
	.\end{align*}
\end{proof}
