\lecture{1}{2020-02-12}{Title}

\setcounter{chapter}{-1}
\chapter{About the Course} \label{chap:about_the_course}
Check toledo. Lots of pdf with more information on there.
Preliminaries
\begin{itemize}
	\item integrals, haar measure
	\item Fourier Transform, abstract over groups
	\item Gauss sums
	\item geometry
		\begin{itemize}
			\item algebraic
			\item differential
		\end{itemize}
	\item logic
\end{itemize}
We will roughly cover the book by neal koblitz.
\section{Three examples of counting/sums} \label{sec:three_examples_of_counting/sums}
\begin{example}
	Let  $f$ be a polynomial in $n$ variables with integer coefficients. 
	Let $N > 0$ be a positive integer.
	\[
		\abs{\left\{ x \in \left( \frac{\Z}{N\Z} \right) ^{n} \; \middle | \; f(x) = 0 \right\}  }
	.\] 
	There also is the finite exponential sum.
	\[
		\sum_{s \in (\Z / N \Z)^{n}} \exp\left( \frac{2\pi i}{N} f(x) \right) 
	.\] 
	The Gauss sum is a basic case where $f(x) = x^2$
	The cases where $N = p^{m}$ are the most important as crt can be use to extend the result to other integers.
	When $n  = p$, prime. We can work over a finite field. Then there is a third counting number.
	 \[
		 \abs{\left\{ x \in (\F_{p^{n}})^n \; \middle | \; f(x) = 0 \right\} }
	.\] 
	Lots of work on this was done by Deligne. He solved the Weil Conjectures.
	This inspired Igusa to formulate similar conjuctures for the other two counting sums.

\end{example}

\chapter{Constructing the $p$-adic numbers} \label{chap:constructing_the_$p$-adic_numbers}
We will construct a new metric on $\Q$ that is more related to modular arithmatic.

\begin{definition}
	Given $\frac{a}{b} \in\Q$. Then we can write $a = p^{l} a', b = p^{k}b'$, with $p \not |\; a', p \not |\; b'$. We define the  $p$-order of $\frac{a}{b}$ to be \[
	\ord_p \frac{a}{b} = \begin{cases}
		\infty  & a = 0 \\
		l - k & x \ne 0
	\end{cases}
	.\] 
\end{definition}
\begin{definition}
	For a prime $p$ the $p$-adic norm is defined as \[
		\abs x _ p = p^{-\ord_p(x)}
	.\] 
\end{definition}
\begin{definition}
	The $p$-adic metric is defined as \[
		d_p(x,y) := \abs{x - y}_p
	.\] 
\end{definition}
\begin{lemma}
	Let $x, y, z \in \Q$ with $|*| = |*|_p$. Then 
	\begin{enumerate}
		\item $\abs{x\cdot y}  = \abs x \cdot  \abs y$
		\item $\abs{x + y} \le \max(\abs x, \abs y)$
	\end{enumerate}
\end{lemma}

