\lecture{1}{2020-02-11}{Inleiding}

\setcounter{chapter}{-1}
\chapter{Over het Vak} \label{chap:over_het_vak}
Les wordt gegeven door Wouter Castryck met oefenzittingen door Robin van der Var.
Er staan cursus nota's op toledo. Examenmoment moet nog vastgelegd worden.

\section{Wat is Discrete wiskunde} \label{sec:wat_is_discrete_wiskunde}
Hoofdstukken 
\begin{enumerate}
	\item Duivenhok princiepe
	\item Polyatheorie: Tel problemen op symmetrie na.
	\item Block designs
	\item Matroid 
	\item Foutenverbeterende codes
	\item Cryptografie
	\item Hypergeometrische sommen
\end{enumerate}

\section{Project} \label{sec:project}
Er komt een programeer projet in Sage. Het telt mee voor 2 punten. Het is verplicht om te maken. 

\chapter{Duivenhok Princiepe} \label{chap:duivenhok_princiepe}
Stel dat we $n+1$ duiven hebben en $n$ hokken dan is er minstens een hok dat twee duiven bevat. Meer wiskundig zouden we dit formuleren als volgt.
\begin{stelling}
	Stel dat $S, S'$ eindige verzamelingen zijn en $\#S > \#S'$. Zij  $f: S \to S'$ een functie. Dan is $f$ niet injectief.
\end{stelling}

\section{Voorbeelden van Toepassingen} \label{sec:voorbeelden_van_toepassingen}
Vaak geeft het duivenhok princiepe mooie argumenten.
Er is het volgende lemma van Erdos.
\begin{lemma}
	Zij $S \subset \{1, 2, \ldots, sn\} $. Stel dat $|S| = n+1$. Dan bestaan er $a, b$ zodat $a |b$.
\end{lemma}
\begin{proof}
	Elk natuurlijk getal $a =2^{l}q$ met $q $ oneven. Voor de getallen in $S$ zijn er apriori slechts $n$ mogelijkheden voor $q$. 
	Wegens het duideven hok princiepe bestaan er $a, b \in S$ zodat $a = 2^l q$ en  $b = 2^k q$. 
	Dus moet  $a | b$ of $b | a$. 
\end{proof}
\begin{lemma}[Erdos-Szekeres]
	Beschouw $a_1, a_2, \ldots, a_{nm + 1} \in \R$ paarsgewijs disjuncte getallen. Veronderstel zvva dat $m \ge n\ge 0$. 
	Er bestaat een strikt stijgende of dalende rij van lengte  $n+1$.
\end{lemma}
\begin{proof}
	We gaan een sterkere uitspraak bewijzen. Er bestaat een strikt stijgende deelrij van lengte $m+1$ of er bestaat een strikt dalende deelrij van lengte  $n + 1$.
	We bewijzen dit uit het ongereimde. 
	Verondestel dat geen van dergelijke rijen bestaat.
	Aan elke index $a_i$ koppelen we $(l_i, l_i')$ waar  $l_i$ de lengte van de langste stijgende rij die eindigt in  $a_i$ en $l_i'$ is de lengte van de lansgte dalende rij die eindigt in $a_i$.
	Uit onze aanname volgt dat  $l_i \le m, l_i' \le  n$. Dus  $(l_i, l_i') \in \{1,\ldots,m\} \times \{1, \ldots, n\} $. 
	Wegens het duivenhok princiepe zijn er dus $i < j$ zodat $(l_i, l_i') = (l_j, l_j')$. 
Maar dit kan niet want als  $a_i < a_j$ dan kunnen we een stijgende rij die eindigd in  $a_i$ uitbreiden met  $a_j$ dus zou $l_j > l_i$. Analoog als  $a_j < a_i$ dan is $l_j' < l_i$. 
\end{proof}

De volgende stelling zou de eerste formele toepassing van DHP zijn.
\begin{stelling}
	[Stelling van dirichlet over diophantische benadering]
	Zij $\alpha \in \R \setminus \Q$. Dan bestaan er oneindig veel $\frac{a}{b} \in \Q$ zodat $|\alpha -\frac{a}{b}| < \frac{1}{b^2}$.
\end{stelling}
\begin{proof}
	Kies $N \ge 2$. Bekijk nu $\{0 \alpha\} , \{1 \alpha\} , \{2 \alpha\} , \ldots, \{N \alpha\} $. Uit het duivenhok princiepe volgt dat er  $\exists i > j$ zodat $\{i \alpha\} , \{ j \alpha\} $ tot hetzelfde deelinterval behoren.
	Dus 
	\begin{align*}
		|(i\alpha  - \left\lfloor i \alpha \right\rfloor - (j\alpha - \left\lfloor j\alpha \right\rfloor ) | &< \frac{1}{N} \\
		|(i -j) \alpha - (\left\lfloor i \alpha \right\rfloor - \left\lfloor j \alpha \right\rfloor)| &< \frac{1}{N}\\
		|b\alpha - a| < \frac{1}{N} \implies |\alpha - \frac{a}{b}| < \frac{1}{Nb} < \frac{1}{b^2}
	.\end{align*}

	We kunnen nu ook bewijzen dat er oneindig veel $\frac{a}{b}$ zijn. Zij $\delta = |\alpha - \frac{a}{b}|$. Kies $N > 1 / \delta$. 
	Passen we het bewijs opniew toe met $N$ dan vinden we een betere (en dus verschillende) benadering.
\end{proof}
\begin{stelling}
	[stelling van Roth]
	Zij $\alpha \in \overline{\Q} \setminus \Q$. Dan zijn er $\forall \epsilon > 0$ slechts eindig veel $\frac{a}{b} \in \Q$ zodat $|\alpha - \frac{a}{b}| < \frac{1}{b^{2+\epsilon}}$
\end{stelling}

\begin{oef}
Zij $S \subset \{1, 2, \ldots, 14\} $. Zij $|S| = 6$. 	
Toon aan dat er twee verschillende deelverzamelingen van $A, B \subset  S$ zodat $\sum_{a \in A} a = \sum_{b \in B}$. 
\end{oef}
 \begin{proof}
Naieve poging:

Er zijn $2^6 -1 =63$ niet lege deelveramelingen van S. De grootst mogelijke som in $9 + 10 + 1+ \ldots + 14 = 69$. Duivenhok princiepe in dit geval is niet naief toepasbaar.

Betere poging 1:

Zij $m = \min S$. De mogelijke sommen behoren tot de range  $m, m + 10 + 11 + \ldots + 14 = m  + 60$. Duivenhok princiepe geeft nu het gevraagde.

Betere poging 2:

Lata $S$ zelf buiten beschouwing. \ldots
\end{proof}

\begin{oef}
	Beschouw een 3 op 7 rooster. Elk vakje is zwart of wit. Toon aan dat er een deelrechthoek bestaat (minstens 2x2) waarvan  de 4 hoekpunten dezelfde kleur hebben.
	(Hint: Kijk naar kleuringen van de kolommen)
\end{oef}
\begin{proof}
Er zijn 8 mogelijke configuraties voor de kolommen
\[
\begin{matrix}
	z \\ z \\z
\end{matrix}, 
\begin{matrix}
	z \\ z \\ w
\end{matrix}, 
\begin{matrix}
z \\ w \\ z	
\end{matrix}
, 
\ldots, 
\begin{matrix}
	w \\ w \\ w
\end{matrix}
.\] 
Merk op dat als twee kolommen hetezelfde patroon hebben dat we de zulke rechthoek kunnen vindenn.
We kunnen devolgende kollommen als hetzelfde hok beschouwen
\[
\left\{
\begin{matrix}
	z \\ z \\w
\end{matrix}
,
\begin{matrix}
	z \\ z \\ z
\end{matrix}
\right\}
, 
\left\{ 
\begin{matrix}
	w \\ w \\ z
\end{matrix},
\begin{matrix}
	w \\ w \\ w
\end{matrix}
\right\} 
, 
\left\{ 
\begin{matrix}
	z \\ w \\z 
\end{matrix}
\right\} 
,
\left\{ 
\begin{matrix}
	w \\ z \\ z
\end{matrix}\right\} ,
\left\{ 
\begin{matrix}
	w \\ z \\ w
\end{matrix}\right\},
\left\{ 
\begin{matrix}
	z \\ w \\ w
\end{matrix}\right\} 
.\] 
\end{proof}

\chapter{Polya-theorie} \label{chap:polya-theorie}
Polya-theorie is het tellen van kleuringenn van een object, rekenign houdend met zijn symmetrie. Als leidraad voor vandaag:
Op hoeveel mannieren kunnen we de vlakken van een cubus kleuren met $m$ kleuren, op rotaties in $\R^3$ na.
\begin{enumerate}
	\item[m = 1]  slechts 1 mannier
	\item[m = 2] 10 mannieren
		\begin{itemize}
			\item 1: alles zwart
			\item 1: 1 wit vlak
			\item 2: 2 witte vlakken
			\item 2: 3 witte vlakken
			\item 2: 4 witte vlakken
			\item 1: 5 witte vlakken
			\item 1: 6 witte vlakken
		\end{itemize}
	\item[m = 3] Heel veel geprusts of gebruik maken van Polya-theorie.
\end{enumerate}

\section{Algemene Opzet}
Zij $X$ een eindige verzameling die we willen kleuren. Zij $K$ een eindige verzameling kleuren.

\begin{definitie}
	Een kleuring is een afbeelding $X \to K$ is een kleuring. De verzameling van alle kleuringen is $K^{X}$.	
\end{definitie}

We noemen $G \subset  \sym(X)$ een groep van symmetrien.

\begin{definitie}
	Een rechtse actie van een groep $G$ op een verzameling $V$ is een afbeelding $V \times  G \to V: (v,g) \mapsto  v \star $, zodat
	\begin{enumerate}
		\item $\forall v \in V: v \star e = v$ 
		\item $\forall v \in V, \forall g_1, g_2: (v \star g_1) \star g_2 = v \star (g_1 g_2)$
	\end{enumerate}
\end{definitie}

\begin{definitie}
	Zij $v \in V$. De orbiet $\orbit(V) = \{v \star g \;|\; g \in G\} \subset  V$.
\end{definitie}

\begin{definitie}
	Zij $v \in V$. De stabilisator van $v$ is \[
		\stab(v)  = \{g \in G \;|\; v \star g = v\} 
	.\] 
\end{definitie}
\begin{theorem}[orbit-stabiliser]
Als  $|G|$ eindig is, dan geldt  $\forall v \in V$ \[
	|G| = |\orbit(v)|\cdot |\stab(v)|
.\] 
\end{theorem}

We kunnen nu ons kleuringsprobleem herformuleren.
Onze groep van symmetrieen $G$ ageert langs rechts op $K^\times $ via $(k, g) \mapsto k \circ g$.

\begin{opmerking}
	Twee kleuringen zijn equivalent als en slechts als ze tot dezelfde orbiet behoren. Ons doel is dus het aantal orbieten bepalen.
\end{opmerking}

\begin{definitie}
	Beschouw $G$ werkend (langs rechts) op $V$. Dan definieren we voor elke $g \in G$  \[
		\fix(g) = \{v \in V \;|\; v \star g = v\} 
	.\] 
\end{definitie}

\begin{stelling}
	[Burnside]
	Zij $G$ een eindige groep werkend (langs rechts) op een eindige verzameling $V$. Dan is het aantal orbieten orbieten gelijk aan \[
		\frac{1}{|G|}\sum_{g \in G}|\fix(g)|
	.\] 
\end{stelling}
