\lecture{2}{2020-02-17}{Nog meer herhaling}
\section{Verband tussen projectieve en affiene meetkunde} \label{sec:verband_tussen_projectieve_en_affiene_meetkunde}
\subsection{Eigenlijke en oneigenlijke punten} \label{sec:eigenlijke_en_oneigenlijke_punten}

\subsection{Van projectieve naar affiene transformaties} \label{sec:van_projectieve_naar_affiene_transformaties}
Stel $\phi = [A]: KP^{n} \to KP^{n}$ is een projectieve tranformatie. 
Stel dat $\phi H_\infty = H_\infty$.
\begin{lemma}
	Tangenti\"ele coordinaten van een hypervlak transfomreren als \[
		(\alpha_0, \ldots, \alpha_n) \mapsto  (\alpha_0, \ldots, \alpha_n) A^{-1}
	.\] 
\end{lemma}
\begin{proof}
	Stel $H \leftrightarrow \alpha_0 x_0 + \ldots + \alpha_n x_n = 0$. Dus we kunnen eisen dat 
	\begin{align*}
		\left[ \begin{pmatrix} x_0 \\ \ldots \\ x_n   \end{pmatrix}  \right]  \in \phi H &\iff \phi^{-1}\left( \left[ \begin{pmatrix} x_0 \\ \ldots \\ x_n \end{pmatrix}  \right]  \right) \in H \\
		&\iff \left[ A^{-1} \begin{pmatrix} x_0 \\ \vdots \\ x_n \end{pmatrix}  \right] \\
		&\iff (\alpha_0, \ldots, \alpha_n) A^{-1} \begin{pmatrix} x_0 \\ \ldots \\ x_n \end{pmatrix} 
	.\end{align*}
\end{proof}

\subsection{Van projectieve naar affiene deelruimten} \label{sec:van_projectieve_naar_affiene_deelruimten}

	\paragraph{Stelling in $\aff^2$}
	De diagonalen van een paralellogram snijden elkaar midden door.

	\paragraph{Overeenkomstige stelling in $\R P^2$}
	Gegeven 4 rechten in $\R P^2$, waarvan geen 3 concurrent zijn (volledige vierzijde).
	Noem de 3 rechten die telkens 2 snijpunten niet, op eenzelfde zijde, verbinden "diagonalen". Op elke diagonaal vormen de snijpunten met de andere diagonalen, samen met 2 hoekpunten op de diagonaal een harmonisch puntenviertal.
\begin{figure}[ht]
    \centering
    \incfig{projectieve-versie-van-parallellogramstelling}
    \caption{projectieve versie van parallellogramstelling}
    \label{fig:projectieve-versie-van-parallellogramstelling}
\end{figure}
\begin{oef}
	Affiene stelling door zijde als $\l_\infty$ te kiezen?	
\end{oef}
