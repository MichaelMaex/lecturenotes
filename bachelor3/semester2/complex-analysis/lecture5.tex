\lecture{5}{2020-03-11}{Spiegeling principe en Residue stelling}

\subsection{Spiegelingprincipe van Schwarz} \label{sec:spiegelingprincipe_van_schwarz}
\begin{stelling}	
Zij $\Omega \open \C^{+} = \{z \in \C \st \im z > 0\} $. 
Zij $f: \Omega \to \C$ holomorf. Neem aan dat $\overline{\Omega}\cap \R = [a, b]$. 
Veronderstel dat $f: \Omega\cup [a, b] \to \C$ is continue uitbreiding en $f(x) \in \R$ voor $x \in [a, b]$.
Dan is er een holomorfe  $F:\Omega \cup \Omega^*\cup [a, b] \to \C$ zodat $F(z) = f(z) $ voor $z \in \Omega$. 
\end{stelling}
\begin{proof}
	Neem \[
		F(z) = \begin{cases}
			f(z) & \text{als }z \in \Omega \cup [a, b] \\
			\overline{f(\overline{z})} & \text{als } z \in \Omega^* 
		\end{cases}
	.\] 
	We willen nu bewijzen dat $F$ holomorf is. Dit kan met de cauchy riemann vergelijkingen, maar dat willen we niet doen.
	Neem $z_0 \in \Omega^*$. Dan $\overline{z_0} \in \Omega$. 
	Dus bestaat er een machtreeks rond $\overline{z_0}$. 
	\[
		f(w) = \sum_{k=1}^{\infty} a_k (w - \overline{z_0}) \text{ in schijf } D_r(\overline{z_0})
	.\] 
	Als $z \in D_r(z_0)$ dans if 
	\begin{align*}
		F(z) &= \overline{f(\overline{z})} = \sum_{k = 1}^{\infty} \overline{a_k}\overline{(\overline{z}- \overline{z_0})^{k}} \\
		     &= \sum_{k = 1}^{\infty} \overline{a_k}(z - z_0)^{k}
	.\end{align*}
	Dus is $F$ holomorf in $\Omega^*$.
	We willen nu bewijzen dat $F$ continue is op $\Omega \cup \Omega^*$
	Neem $z_k \to X \in [a, b]$, met  $z_k \in \Omega^*$. Dan is  $F(z_k) = \overline{f(\overline{z_k})}$.
	Dan $\overline{z_k} \in \Omega$ en $f(\overline{z_k} \to f(x)$. We weten dat $f(x)$ reel is.

	Ten slotte willen we aantonen dat $F$ holomorf is op $[a,b]$. 
	We bewijzen dit met Morera.
\end{proof}

\chapter{Meromorfe Functies en het Logaritme} \label{chap:meromorfe_functies_en_het_logaritme}
\section{Nulpunten en polen} \label{sec:nulpunten_en_polen}
\begin{herhaling}
Zij $\Omega$ open en $f:\Omega \to \C$ is holomorf. 
Zij $z_0 \in \Omega$ als $f(z_0) = 0$. Veronderstel dat $f$ niet constant is in een omgeving van $z_0$. 
Dan is $z_0$ een ge\"isoleerd punt.	
\end{herhaling}

\begin{definitie}
	Zij $\Omega$ open $z_0 \in \Omega$ en $f: \Omega \setminus \{z_0\} \to \C$ holomorf.
	Dan is $z_0$ een \emph{ge\"isoleerde singulariteit} van $f$. 
	We kunnen een aantal gevallen onderscheiden.
	\begin{enumerate}
		\item \emph{Ophefbare singulariteit} als er een $w_0$ is zodat \[
		z \in \Omega \mapsto  \begin{cases}
			f(z) & \text{als }z \ne z_0 \\
			w_0 & \text{als } z = z_0
		\end{cases}
		\]
		holomorf is.
	\item $f$ heeft een \emph{pool} is $z_0$ als het geen ophefbare sigulariteit heeft en $f$ geen nulpunten heeft van $z_0$ en $\frac{1}{f}$ heeft een ophefbare sigulariteit. 
		Dan heeft $\frac{1}{f}$ een nulpunt in $z_0$ dus is de orde van pool van $f $ in $z_0$. Stel $n$ is de orde van dat nulpunt. Dan is $ n$ de orde van de pool. 

	\item $f$ heeft een \emph{essentiele singulariteit} in $z_0$ als de singulariteit niet ophefbaar is en ook geen pool is.
	\end{enumerate}
\end{definitie}
\begin{vb}
	De functie \[
		e^{\frac{1}{z}} = \sum_{k = -\infty}^{0} \frac{1}{(-k)!}z^{k}	
	\]
	heeft een essenti\"ele sigulariteit in $z =  0$.
\end{vb}


Stel dat $f$ een pool heeft in $z_0$ en dat \[
	f(z) = \sum_{k = -n}^{\infty}c_k (z - z_0)^{k}
.\] 
Dan is $c _{-1}$ het residu van $f$ in $z_0$. 

\begin{stelling}
	[Residu stelling]
	Zij $\Omega$ open $f$ holomorfh op $\Omega \setminus \{ z_1, z_2, \ldots, z_k\} $ met polen in $z_1, \ldots, z_k \in \Omega$. 
	Zij $\gamma$ een gesloten contour in $\Omega$ met $z_j \in \gamma$ voor $j = 1 \ldots k$. 
	Het gebied omsloten door $\gamma$ behoort tot  $\Omega$. 
	Dan is \[
		\oint_\gamma f(z) \dd z  = 2\pi i \sum_{z_j \text{ binnen countour}} \text{Res}_{z_j}(f)
	.\] 
\end{stelling}
\begin{quote}
	\textit{De residue stelling maakt van integreren algebra, wat veel makkelijker is dan analyse. - Arno Kuijlaars}
\end{quote}
\begin{proof}
	Zij $\tilde \gamma$ dat gebied omsluit waar $f$ holomorf is. 
	Volgens de afbeelding
	\begin{align*}
		\int_{\tilde \gamma} f(z) \dd  z &= 0 \\
						 &= \oint_{\gamma}f(z)\dd z - \oint_{c_r(z_1)} f(z) \dd z - \oint_{c_r(z_2)} - \oint_{c_r(z_3)}f(z) \dd z
	.\end{align*}
\end{proof}
\begin{figure}[ht]
    \centering
    \incfig{residue-stelling}
    \caption{residue stelling}
    \label{fig:residue-stelling}
\end{figure}

\begin{vb}
	We willen \[
		\int_{0}^{\infty} \frac{x^{\alpha}}{(1 + x)^2} \dd x
	\] uitrekenen voor $-1 < \alpha < 1$.
	Wat bedoelen we met $z^{\alpha}$? We bedoelen $e^{\alpha \ln z}$. Wat is $\ln z$?
	 \[
		 \ln(z) = \ln \abs z + i \mathrm{arg}(z)
	.\] 
	Hoe kiezen $\mathrm{arg}$? We kunen het tussen  $-\pi, \pi$ kiezen. Nu kiezen we tussen  $0 $ en $2\pi$.
	met deze keuze is  $f$ meromorf een gebied dat $\R^{+}$ niet bevat.
	Als $\delta \to 0$ vallen de twee rechten niet weg. Nee de functie is daar niet eens continue. We kunne wel het verschil van de bijdragen berekenen. 
	Dus 
	\begin{align*}
		\int_{\epsilon}^{R} \frac{x^{\alpha}}{(1 + x)^2} \dd x + \int_{R}^{\epsilon }\frac{x^{\alpha} e^{i\alpha 2 \pi }}{( 1 + x)^2} \dd x &= (1 - e^{i\alpha 2 \pi}) \int_{\epsilon}^{R} \frac{x^{\alpha}}{(1 + x)^2 } \dd x 
	.\end{align*}

	ML afschatting geeft dat de integraal over  $C_{\epsilon}(0)$ naar nul gaat.
	Dus \[
		(1 - e^{i \alpha 2 \pi}) \int_{0}^{\infty}\frac{x^{\alpha}}{(1 + x)^2} \dd x	 = 2 \pi \mathrm{Res}_{-1}f
	.\] 

\end{vb}
\begin{figure}[ht]
    \centering
    \incfig{contour-logaritme}
    \caption{contour logaritme}
    \label{fig:contour-logaritme}
\end{figure}
