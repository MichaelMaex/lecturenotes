\lecture{7}{2020-03-25}{Homotopien en enkelvoudig samenhangend}

\section{Homotopien en enkelvoudig samenhangende domeinen} \label{sec:homotopien_en_enkelvoudig_samenhangende_domeinen}
\begin{definitie}
	[homotopie]
	Zij $\Omega$ open in $\C$. 
	Zij $\gamma_0, \gamma_1$ twee krommen van $a$ naar $b$ in $\Omega$. 
	Dan $\gamma_0, \gamma_1$ homotoop als er een continue functie  \begin{align*}
		h: [0,1] \times [0,1] &\longrightarrow \Omega 
	.\end{align*}
	Zodat $H(0, t) = \gamma_0(t)$ en $H(0, 1) \gamma_1(t)$ en $H(s, 0) = a, H(s, 1) = b$. 
\end{definitie}
We definieren $\gamma_s: [0, 1] \to \Omega: t\mapsto H(s, t)$ een kromme van $a$ naar $b$.

\begin{definitie}
	$\Omega$ is enkelvoudig samenhangend als $\Omega$ samenhangend is en elke twee krommen $\gamma_0, \gamma_1 $ met zelfde begin en eind homotoop zijn.
\end{definitie}
\begin{vb}
	Een schijf is enkelvoudig samenhangend. 
	$\C \setminus (-\infty, o] $ is enkelvoudig samenhangend. 
	$D \setminus \{0\} $ is niet enkelvoudig samenhangend.
\end{vb}

\begin{stelling}
	Zij $\Omega$ open en $f: \Omega \to \C$ holomorph. 
	Neem aan $\gamma_0$ en $\gamma_1$ zijn homotoop. 
	Dan is \[
	\int_{\gamma_0}^{} f \dd z = \int_{\gamma_1}^{}  f \dd z  
	.\] 
\end{stelling}

\begin{figure}[ht]
    \centering
    \incfig{homotope-krommen}
    \caption{homotope krommen}
    \label{fig:homotope-krommen}
\end{figure}
\begin{proof}
Zie tekening. Neem primitieven op schijven $D_i$. 
\end{proof}
