\lecture{1}{2020-02-10}{Introductie en Restklasse ringen}
\setcounter{chapter}{-1}
\chapter{Praktische info}
\begin{enumerate}
	\item Communicatie via Toledo
	\subitem cursustekst, oefeningen, contact gegevens
\item oefenzittingen starten vanaf donderdag 20 februari
\item Examenvorm
	\begin{itemize}
		\item schriftelijk examen in de examenperiode (geen mondeling deel, gesloten boek, accent op oefening).
		\item 3 facultatieve taken; tellen enkel mee indien voordelig.
	\end{itemize}
\end{enumerate}

\section{Inhoud} \label{sec:inhoud}
Inleiding tot de algebraische en analytische getaltheorie.
\begin{enumerate}
	\item Structuur van de ringen $\Z / n \Z$. 
	\item Quadratic reciprocity!!!!!
	\item Priemtesten
	\item p-adische getallen
	\item Princiepe van Hasse, Rationale punten op Kegelsneden.
	\item Priemgetallen in aritmetische rijen.
		Stelling van Dirichlet. 
	\item Elliptic curves (als er tijd over is). 
\end{enumerate}

\chapter{Achtergrond} \label{chap:achtergrond}

\begin{definitie}
	Zij $G$ een eindige groep. De exponent $\exp(G)$ is het kleinste gemene veelvoud van de elementen van $G$. 
\end{definitie}
Uit de stelling van lagrange volgt dat $\exp(G) \;|\; \#G$.
\begin{vb}
	$\Z / 4\Z , + )$ heeft exponent 4. $(\Z / 2 \Z, +) ^2$ heeft exponent 2. 
\end{vb}
Als $G$ cyclisch is than is $\exp G = \#G$. Het omgekeerde is waar als $G$ abels is.

