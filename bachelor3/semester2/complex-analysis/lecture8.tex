\lecture{8}{2020-04-01}{Gamma functie}
\section{Gamma en Zeta Functie} \label{sec:gamma_en_zeta_functie}

\begin{definitie}
	De \emph{gamma functie} is gedefinieerd adhv het volgende voorschrift \[
		\Gamma(z) = \int_{0}^{\infty} e^{-t}t ^{z -1} \dd t, \re z > 0 
	.\] 
\end{definitie}
\begin{propositie}
	$F(z)$ is holomorf voor $\re z > 0$.
\end{propositie}
\begin{proof}
	Neem $0 < \epsilon < R < \infty$. 
	Bekijk  
	\[
		f_{\epsilon, R}(z) = \int_{\epsilon}^{R} e^{-t}t ^{z-1} \dd t 
	.\]
	Dit is een gehele functie (Morera).
	We willen nu bewijzen dat \[
		\lim_{\substack{\epsilon \to 0 \\ R \to \infty}}  f_{\epsilon, R} (z) = \Gamma(z)
	\] 
	uniform op $0 < \delta < \re z < M < \infty$. 

	\begin{align*}
		\abs{ \Gamma(z) - f_{\epsilon, R}(z)} &= \abs{\left( \int_{0}^{\epsilon} + \int_{R}^{\infty} e^{-t}^{z-1}\dd t   \right) } \\
		\abs{\int_{0}^{\epsilon} e^{-t}t ^{z -1} \dd t}  &\le \int_{0}^{\epsilon} t ^{\re z - 1} \dd t , \epsilon < 1 \\
								 &\le \int_{0}^{\epsilon} t ^{\delta -1} \dd t \to 0 , \text{ als } \epsilon \to 0 \\
		\abs{\int_{R}^{\infty} e^{-t}t ^{z - 1}\dd t} &\le \int_{R}^{\infty} e^{-t} t ^{\re z - 1} \dd t \\
							      &\le \int_{R}^{\infty} e^{-t} t ^{M -1} \dd t \to 0 \text{ als } R \to \infty 
	.\end{align*}


\end{proof}
\begin{eigenschap}
\begin{enumerate}
	\item $\Gamma(1) = \int_{0}^{\infty} e^{-t}dt = 1 $ 
	\item $\Gamma(2) = 1 $
	\item $\Gamma(n + 1) = n!$
	\item Functionaal vergelijking \[
			z \Gamma(z) = \Gamma(z + 1), \re z > 0
	.\] 
\end{enumerate}		
\end{eigenschap}
\begin{proof}
\begin{align*}
	z \Gamma(z) &= \int_{0}^{\infty} e^{-t}z t ^{z - 1} \dd t \\
		    &= \int_{0}^{\infty} e^{-t} \dd t ^{z}  \\
		    &= \int_{0}^{\infty} e^{-t}t ^{z} \dd t = \Gamma (z + 1)  
.\end{align*}	
\end{proof}
\begin{stelling}
	$\Gamma$ heeft een mermorphe uitbreiding tot $\C$ met polen in $0, -1, -2, -3, \ldots$. 
\end{stelling}
\begin{proof}
	Bekijk $F_1(z) = \Gamma(z + 1) / z$ voor $\re z > -1$.
	Dan is  $F_1(z) = \Gamma(z)$ als $\re z > 0$. En $F_1(z) $ is gedefinieerd voor $\re z > -1$. 
	Ook geld \[
		z F_1(z) = F_1(z + 1)
	.\] wegens de identiteits stelling. 
	Dit blijven we herhalen\ldots
\end{proof}
\begin{stelling}
	[Spiegelingsprinciepe voor gamma functie]
	\[
		\Gamma(z) \Gamma(1 - z) = \frac{\pi}{\sin \pi z}, \text{ voor } z \in \C
	.\] 
\end{stelling}
\begin{proof}
	We bewijzen het voor $(0, 1)$,$0< z = x < 1$. Dan geeft de identiteitststelling dat het overal gelijk is.
	Dan 
	\begin{align*}
		\Gamma(x) \Gamma(1 - x) &= \int_{0}^{\infty} e^{-s} s^{x  - 1} \dd s \int_{0}^{\infty} e^{-t} t ^{-x} \dd t  \\
		&= \int_{0}^{\infty} \int_{0}^{\infty} e^{-s-t}s^{x-1}t ^{-x}\dd s \dd t   
	.\end{align*}
	We voeren nieuwe coordinaten in 
	\begin{align*}
		s &= r \cos^2 \theta\\
		t &= r \sin^2 \theta
	.\end{align*}
	De jacobiaan is $2r \cos \theta \sin \theta$. 
	Dan wordt de integraal 
	\begin{align*}
		&\int_{0}^{\infty} \int_{0}^{\frac{\pi}{2}} e^{-r}(r\cos^2\theta)^{x-1}(r \sin^2\theta)^{-x}2r \cos \theta \sin \theta \dd \theta \dd r \\
		&= 2 \int_{0}^{\infty} e^{-r}\dd r \int_{0}^{\frac{\pi}{2}} (\cos \theta)^{2x -1}(\sin \theta)^{-2x+1} d \theta   \\
		&= 2 \int_{0}^{\frac{\pi}{2}} (\tan \theta)^{-2x} \dd \theta  
	.\end{align*}
	We doen opnieuwe een substitutie.
	\begin{align*}
		\tan \theta &= u \\
		\theta &=  \arctan u  \\
		\dd \theta &= \frac{1}{1 + u^2} \dd u
	.\end{align*}
	Dan vinden we 
	\begin{align*}
		2 \int_{0}^{\frac{\pi}{2}} (\tan \theta)^{1-2x} \dd \theta &= 2 \int_{0}^{\infty} \frac{u^{1 - 2x}}{1 + u^2} \dd u \\
									   &= 2 \frac{\pi}{2 \cos\left( \frac{(1 - 2x)\pi}{2} \right) } \\
									   &= \frac{\pi}{\sin(\pi x)} 
	.\end{align*}
\end{proof}
\subsection{Zetafunctie} \label{sec:zetafunctie}
\begin{definitie}
	De zeta functie is \[
		\zeta (s) = \sum_{n=1}^{\infty} \frac{1}{n^{s}}, \re s > 1
	.\] 
	Deze functie is holomorph.
\end{definitie}
Een belangrijke eigenschap is
\[
	\zeta(s) = \prod_{p \text{ priem}}^{} (1 - p^{-s})^{-1} 
.\] 

\begin{stelling}
	[prime number theorem]
	\[
		\pi(x) = \# \{\text{priem } p \le x\} 
	.\] 
	Dan is \[
		\pi(x) \sim \frac{x}{\log x} \text{ als } x \to \infty
	.\] 
\end{stelling}
Er is een analystische voorzetting van de $\zeta$-functie tot $\C \setminus \{1\} $. 
Dit tonen we aan met een hulpfunctie  \[
	\xi (s) = \pi^{-s x/2}\Gamma\left(\frac{s}{2}\right) \zeta(s)
.\] 
Deze heeft een functie vergelijking
\[
	\zeta(s) = \xi(1-s)
.\]
Dan heeft de $\zeta$ functie nulpunten in $s = -2, -4, -6, \ldots$ . Dit zijnn triviale nulpunten.
De niet trivaiale nulpunten bevinden zich tussen $0 \le \re s \le 1$. 
Niet triviale nulpuntenn bevindenn zich in $0 < \re S < 1$ impliceert de prime number theorem.
